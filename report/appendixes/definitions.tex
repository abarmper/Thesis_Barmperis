\chapter{Ορισμοί Εννοιών}

Το παρόν παράρτημα περιέχει ορισμούς εννοιών που εισάγονται κατά τη διάρκεια της παρούσας εργασίας. Κατά αυτόν τον τρόπο, δε διακόπτεται η ροή του κυρίως κειμένου.

\begin{description}
    \item[Τεχνητή Νοημοσύνη] \hfill \\ 
    Έχουν υπάρξει πολλοί διαφορετικοί ορισμοί της Τεχνητής Νοημοσύνης: Μερικοί την περιγράφουν σαν εσωτερική διαδικασία της σκέψης που προσομοιάζει αυτή του ανθρώπου ενώ άλλοι ως εξωτερική διαδικασία μαθηματικά βέλτιστης συμπεριφοράς. Σύμφωνα με το κυρίαρχο μοντέλο, η Τεχνητή Νοημοσύνη ασχολείται κυρίως με τη λογική δράση. Ένας ιδανικός ευφυής πράκτορας δρα βέλτιστα σε κάθε περίσταση. Έτσι λοιπόν, η μελέτη της δημιουργίας ευφυών πρακτόρων μπορεί να τεθεί ως ορισμός της Τεχνητής Νοημοσύνης.
    \item[Μηχανική Μάθηση] \hfill \\ 
    Με λίγα λόγια, πρόκειται για τον κλάδο της τεχνητής νοημοσης ο οποίος ασχολείται με την ανάπτυξη υπολογιστικών συστημάτων ικανών να μαθαίνουν από παραδείγματα. Αναλυτικότερα, μπορούν και προσαρμόζονται χωρίς να ακολουθούν ρητές εντολές αλλά μέσω αλγορίθμων και στατιστικών μοντέλων που τους επιτρέπουν να αναλύουν και να εξάγουν συμπεράσματα μοτίβα σε δεδομένα. Χαρακτηριστικό γνώρισμα των συστημάτων μηχανικής μάθησης είναι η ικανότητά τους να βελτιώνουν την απόδοσή τους σε μια εργασία (όπως αυτή μετράται με κάποια κατάληλη μετρική) όσο η \textquote{εμπειρία} τους σε αυτήν την εργασία αυξάνεται.
    \item[Τεχνητά Νευρωνικά Δίκτυα] \hfill \\ 
    Τα τεχνητά νευρωνικά δίκτυα αποτελούν ένα αλγοριθμικό κατασκεύασμα από απλούς υπολογιστικούς κόμβους διασυνδεδεμένους μεταξύ τους μέσω ακμών κάτω από μια συγκεκριμένη τοπολογία. Εμπνευσμένα από τα βιολογικά νευρωνικά δίκτυα, οι κόμβοι μπορούν να παρομοιαστούν με κύτταρά νευρώνων ενώ οι ακμές με νευρικές συνάψεις. \\
    Τα τεχνητά νευρωνικά δίκτυα είναι παράδειγμα συστήματος μηχανικής μάθησης αφού μετά την κατάλληλη εκπαίδευσή τους, γενικεύουν από τα δεδομένα (\en{inference}) και τελικά υπό μια αφαιρετική σκοπιά, αποτελεί το καθένα μια συνάρτηση που αντιστοιχίζει δεδομένα από τον χώρο εισόδου σε \textquote{προβλέψεις} του χώρου εξόδου.
    \item[Βαθιά Μάθηση] \hfill \\ A short one-line description.
    \item[Γνωστική Νευρoεπιστήμη] \hfill \\ A short one-line description. 
    \item[Ταξινόμηση Προτύπων] \hfill \\ A short one-line description.  
    \item[Γραμμικά Διαχωρίσιμες Κλάσσεις] \hfill \\ A short one-line description.
    \item[Γραμμικά Μοντέλα] \hfill \\ A short one-line description.   
    \item[Γενετικοί Αλγόριθμοι] \hfill \\ A short one-line description.
    \item[Επιβλεπόμενη Μάθηση] \hfill \\ A short one-line description.  
    \item[Μη-επιβλεπόμενη Μάθηση] \hfill \\ A short one-line description.
    \item[Ενισχυτική Μάθηση] \hfill \\ A short one-line description.    
 \end{description}