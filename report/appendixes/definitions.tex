\chapter{Ορισμοί Εννοιών}
\label{chap:definitions}
Το παρόν παράρτημα περιέχει ορισμούς εννοιών που εισάγονται κατά τη διάρκεια της παρούσας εργασίας. Κατά αυτόν τον τρόπο, δε διακόπτεται η ροή του κυρίως κειμένου. \cite{russell2020artificial,goodfellow2016deep,geron2019hands, bishop2006pattern} 

\begin{description}
    \item[Τεχνητή Νοημοσύνη] \hfill \\ 
    Έχουν υπάρξει πολλοί διαφορετικοί ορισμοί της Τεχνητής Νοημοσύνης: Μερικοί την περιγράφουν σαν εσωτερική διαδικασία της σκέψης που προσομοιάζει αυτής του ανθρώπου ενώ άλλοι ως εξωτερική διαδικασία μαθηματικά βέλτιστης συμπεριφοράς. Σύμφωνα με το κυρίαρχο μοντέλο, η Τεχνητή Νοημοσύνη ασχολείται κυρίως με τη λογική δράση. Ένας ιδανικός ευφυής πράκτορας δρα βέλτιστα σε κάθε περίσταση. Έτσι λοιπόν, η μελέτη της δημιουργίας ευφυών πρακτόρων μπορεί να τεθεί ως ορισμός της Τεχνητής Νοημοσύνης.
    \item[Μηχανική Μάθηση] \hfill \\ 
    Με λίγα λόγια, πρόκειται για τον κλάδο της τεχνητής νοημοσύνης ο οποίος ασχολείται με την ανάπτυξη υπολογιστικών συστημάτων ικανών να μαθαίνουν από παραδείγματα. Αναλυτικότερα, μπορούν και προσαρμόζονται χωρίς να ακολουθούν ρητές εντολές αλλά μέσω αλγορίθμων και στατιστικών μοντέλων που τους επιτρέπουν να αναλύουν και να εξάγουν συμπεράσματα από μοτίβα σε δεδομένα. Χαρακτηριστικό γνώρισμα των συστημάτων μηχανικής μάθησης είναι η ικανότητά τους να βελτιώνουν την απόδοσή τους σε μια εργασία (όπως αυτή μετράται με κάποια κατάλληλη μετρική) όσο η \textquote{εμπειρία} τους σε αυτήν αυξάνεται \cite{mitchell1997machine}.
    \item[Τεχνητά Νευρωνικά Δίκτυα] \hfill \\ 
    Τα τεχνητά νευρωνικά δίκτυα αποτελούν ένα αλγοριθμικό κατασκεύασμα από απλούς υπολογιστικούς κόμβους διασυνδεδεμένους μεταξύ τους μέσω ακμών κάτω από μια συγκεκριμένη τοπολογία (συνήθως οργανώνονται σε επίπεδα, βλ. \ref{sec:_vanilla_nn}). Εμπνευσμένα από τα βιολογικά νευρωνικά δίκτυα, οι κόμβοι μπορούν να παρομοιαστούν με κύτταρα νευρώνων ενώ οι ακμές με νευρικές συνάψεις. \\
    Τα τεχνητά νευρωνικά δίκτυα είναι παράδειγμα συστήματος μηχανικής μάθησης αφού μετά την κατάλληλη εκπαίδευσή τους, γενικεύουν από τα δεδομένα (\en{inference}). Τελικά, μετά την ανάπτυξή τους, υπό μια αφαιρετική σκοπιά αποτελεί το καθένα μια συνάρτηση που αντιστοιχίζει δεδομένα από τον χώρο εισόδου σε \textquote{προβλέψεις} του χώρου εξόδου.
    \item[Βαθιά Μάθηση] \hfill \\ Αποτελεί μια υποκατηγορία μηχανικής μάθησης όπου χρησιμοποιούνται πολυεπίπεδα νευρωνικά δίκτυα. Τα πολλαπλά επίπεδα που διαθέτουν τους επιτρέπουν να μαθαίνουν και να αναγνωρίζουν εσωτερικά, γενικευμένα χαρακτηριστικά των δεδομένων εισόδου.
    \hypertarget{_computational_neuroscience}{
    \item[Υπολογιστική Νευροεπιστήμη (\en{Computational Neuroscience})]} \hfill \\Πρόκειται για τον κλάδο της Νευρωεπιστήμης που χρησιμοποιεί μαθηματικά μοντέλα, μαθηματική ανάλυση και προσεγγιστικά προς τον εγκέφαλο συστήματα για να κατανοήσει τις αρχές ανάπτυξης, δομής, φυσιολογίας καθώς και των γνωστικών (\en{cognitive}) ικανοτήτων του νευρικού συστήματος.
    \item[Επιβλεπόμενη Μάθηση] \hfill \\ Στους αλγορίθμους επιβλεπόμενης μάθησης, ως είσοδος παρέχεται ένα σύνολο δεδομένων μαζί με τους επιθυμητούς στόχους. Δηλαδή, τα δεδομένα δίνονται σε ζεύγη (παράδειγμα εισόδου\textemdash επιθυμητή τιμή εξόδου). Με βάση αυτά, το σύστημα καλείται να εξάγει μια συνάρτηση η οποία θα έχει μάθει να μοντελοποιεί τη σχέση εισόδου\textendash εξόδου μέσα από τα παραδείγματα και τελικά θα είναι ικανή να προβλέψει την τιμή εξόδου σε νέα παραδείγματα για τα οποία η τιμή στόχος είναι άγνωστη. Συνήθως, εκτός από τα δεδομένα για την εκπαίδευση υπάρχουν και άλλα σύνολα δεδομένων για τον έλεγχο της απόδοσης του συστήματος πρόβλεψης.\\
    Ανάλογα με το αν η τιμή στόχος είναι διακριτή ή συνεχής, έχουμε αντίστοιχα το πρόβλημα ταξινόμησης (\en{classification}) ή της παλινδρόμησης (\en{regression}). Παράδειγμα συστήματος ταξινόμησης επιβλεπόμενης μάθησης είναι το φίλτρο ανεπιθύμητης αλληλογραφίας το οποίο αφού εκπαιδεύτηκε με ένα σύνολο επισημασμένων αλληλογραφιών ως ανεπιθύμητων ή επιθυμητών έμαθε να εντοπίζει νέα εισερχόμενη ανεπιθύμητη αλληλογραφία. Ένα παράδειγμα συστήματος παλινδρόμησης επιβλεπόμενης μάθησης είναι αυτό της πρόβλεψης τιμών μετοχών καθώς ο στόχος (κόστος μετοχής) είναι συνεχής αριθμός.

    \item[Μη-επιβλεπόμενη Μάθηση] \hfill \\ Οι αλγόριθμοι μη\textendash επιβλεπόμενης μάθησης, σε αντιδιαστολή με τους αλγορίθμους επιβλεπόμενης μάθησης, δέχονται ως είσοδο ένα σύνολο δεδομένων που περιλαμβάνει παραδείγματα, χωρίς όμως να συνοδεύονται από αντίστοιχες τιμές\textendash στόχους. Στην περίπτωση αυτή, το υπό εκπαίδευση σύστημα επιχειρεί να μάθει πρότυπα στα δεδομένα εισόδου χωρίς κάποιο μηχανισμό ανατροφοδότησης. Συνήθεις εφαρμογές μη\textendash επιβλεπόμενης μάθησης είναι αυτές της ομαδοποίησης των δεδομένων σε συστάδες ή της αναπαράστασής τους με ένα γράφημα.
    
    \item[Ενισχυτική Μάθηση] \hfill \\ Στην ενισχυτική μάθηση, στο σύστημα (το οποίο καλείται \textquote{ευφυής πράκτορας} στο πλαίσιο αυτό) δεν παρέχεται κάποιο σύνολο δεδομένων αλλά η όποια εμπειρία αποκτάται μέσω της αλληλεπίδρασής του με το περιβάλλον. Ο πράκτορας έχει τη δυνατότητα να παρατηρήσει το περιβάλλον του και τη (πιθανή) κατάστασή του και ανάλογα με μια στρατηγική (\en{policy}) να δράσει σε αυτό. Το περιβάλλον του, με κάθε δράση (και ανάλογα την κατάσταση) παρέχει την απαραίτητη εμπειρία υπό τη μορφή επιβράβευσης (\en{reward}) ή ποινής (\en{punishment}). Έτσι, ο πράκτορας μαθαίνει από την εμπειρία προσαρμόζοντας τη στρατηγική του ώστε να μεγιστοποιεί την επιβράβευση την οποία λαμβάνει και τελικά να πετυχαίνει τον στόχο του. Παράδειγμα ενός τέτοιου πράκτορα είναι ένα σύστημα το οποίο παίζει σκάκι.

    \item[Μάθηση Κατά Δέσμες] \hfill \\ Αφορά το είδος συστημάτων μηχανικής μάθησης που δεν έχουν τη δυνατότητα να μαθαίνουν σταδιακά αλλά εκπαιδεύονται μονομιάς χρησιμοποιώντας όλο το σύνολο δεδομένων στην είσοδό τους. Σε περίπτωση που προστεθούν νέα δεδομένα στα οποία θα επιθυμούσαμε το σύστημα να προσαρμοστεί, απαιτείται εκ νέου εκπαίδευση στο καινούριο σύνολο δεδομένων το οποίο θα περιέχει τόσο τα παλαιά όσο και τα επιπρόσθετα δεδομένα (διαδικασία χρονοβόρα και υπολογιστικά κοστοβόρα). Συνήθως, σε τέτοιες περιπτώσεις το σύστημα πρέπει να σταματήσει να λειτουργεί και να μεταβεί στη φάση σχεδιασμού. Παραδείγματα αυτών των μεθόδων αποτελούν ο αλγόριθμος \en{Expectation Maximization} και ο \en{Self-organizing map} όπως περιγράφεται στην ενότητα \ref{sec:_SOM}.
    
    \item[Μάθηση σε Ζωντανό Χρόνο] \hfill \\ Πρόκειται για τα συστήματα μηχανικής μάθησης που, σε αντίθεση με αυτά που μαθαίνουν κατά δέσμες, είναι ικανά να εκπαιδεύονται σταδιακά, είτε με ένα παράδειγμα τη φορά είτε με μικρές δέσμες παραδειγμάτων στην είσοδό τους. Το θετικό σε αυτά τα συστήματα είναι η δυνατότητα προσαρμογής τους σε νέα δεδομένα με πολύ μικρό χρονικό και υπολογιστικό κόστος. Αποτέλεσμα αυτού είναι ότι υπάρχει (συνήθως) η δυνατότητα η εκπαίδευσή τους να γίνει ζωντανά (\en{online}) χωρίς να σταματήσει η λειτουργία του συστήματος. Παράδειγμα αποτελούν οι εφαρμογές πρόβλεψης τιμών μετοχών όπου απαιτείται συνεχής προσαρμογή του συστήματος στα νέα δεδομένα της αγοράς. 
    
    \item[Μάθηση Βασισμένη σε Παραδείγματα] \hfill \\ Είναι μια οικογένεια απλών συστημάτων μηχανικής μάθησης που αφορά τον τρόπο με τον οποίο ένα σύστημα γενικεύει από τα παραδείγματα του συνόλου εισόδου. Στα συγκεκριμένα, όταν τα τροφοδοτούμε με κάποιο νέο παράδειγμα, το συγκρίνουν με τα δεδομένα εισόδου (ή ένα υποσύνολο αυτών) τα οποία έχουν αποθηκευθεί στη μνήμη τους κατά την εκπαίδευση. Ένα χαρακτηριστικό μειονέκτημα αυτών των συστημάτων είναι ότι ο χώρος που απαιτείται για την αποθήκευση του μοντέλου (του συστήματος μάθησης μετά την εκπαίδευσή του) αυξάνεται με το μέγεθος του συνόλου εισόδου (συνήθως με γραμμικό τρόπο). Ενδεικτικά, ένα σύστημα που γενικεύει κατά αυτόν τον τρόπο είναι το \en{K-nearest neighbors}.
    
    \item[Μάθηση Βασισμένη σε μοντέλο] \hfill \\ Είναι μια άλλη οικογένεια συστημάτων όπου η μηχανική μάθηση γίνεται μέσω της προσαρμογής (\en{fitting}) ενός μοντέλου στα δεδομένα εισόδου. Έχοντας εκφράσει το σύνολο των δεδομένων εκπαίδευσης (ή τη σχέση αυτών με την επιθυμητή έξοδο) χρησιμοποιώντας ένα κατάλληλα εκφραστικό (\en{expressive}) μοντέλο, λέμε ότι το σύστημα μαθαίνει να \textquote{γενικεύει} από τα παραδείγματα. Έτσι, για να παράξει προβλέψεις σε νέα δεδομένα, δεν απαιτείται η αποθήκευση όλων των δεδομένων εκπαίδευσης αλλά μόνο των παράμετρων του μοντέλου που εκφράζει.

    \item[Γνωστική Νευρoεπιστήμη] \hfill \\ Η Γνωστική νευρωεπιστήμη είναι το πεδίο μελέτης που ασχολείται με τα νευρωνικά υποστρώματα των διανοητικών διεργασιών. Είναι η τομή της ψυχολογίας με τη νευροεπιστήμη. Συνδυάζει τις θεωρίες της γνωστικής ψυχολογίας και της υπολογιστικής μοντελοποίησης με πειραματικά δεδομένα του εγκεφάλου.
    
    \item[Αναγνώριση Προτύπων] \hfill \\ Είναι ένα επιστημονικό πεδίο με στόχο την ανάπτυξη αλγορίθμων για την αυτοματοποιημένη απόδοση κάποιας τιμής (παλινδρόμησης) ή διακριτικού στοιχείου (ταξινόμηση) με βάση μοτίβα/χαρακτηριστικά που παρατηρούνται στα εισαγόμενα δεδομένα, συνήθως κωδικοποιημένα ως αλληλουχίες αριθμών.  
    \item[Γραμμικά Διαχωρίσιμες Κλάσεις] \hfill \\ Λέμε ότι ένα σύνολο δεδομένων για ταξινόμηση που περιέχει δύο κλάσεις είναι γραμμικά διαχωρίσιμο αν και μόνο αν μπορούμε να διαχωρίσουμε τις δύο κλάσεις στον πολυδιάστατο χώρο χαρακτηριστικών εισόδου χρησιμοποιώντας ένα υπερεπίπεδο. Στην περίπτωση όπου ο χώρος χαρακτηριστικών είναι δισδιάστατος, αρκεί να μπορούμε να χαράξουμε μια ευθεία γραμμή στο καρτεσιανό επίπεδο που να διαχωρίζει τις δύο κλάσεις.
    
    \item[Γραμμικά Μοντέλα] \hfill \\ Τα γραμμικά μοντέλα περιγράφουν τη σχέση μεταξύ ενός ή περισσοτέρων μεταβλητών εισόδου (μεταβλητές πρόβλεψης) και μιας συνεχούς τιμής εξόδου (απόκρισης). Η χρήση των μοντέλων αυτών ενδείκνυται όταν οι σχέσεις μεταξύ εισόδου\textendash εξόδου είναι (σχεδόν) γραμμικές στο διάστημα μελέτης. Μια στατιστική μέθοδος για την παραγωγή γραμμικών μοντέλων που μοντελοποιούν αυτές τις σχέσεις από σύνολα δεδομένων εισόδου\textendash εξόδου είναι η γραμμική παλινδρόμηση.
     
    \item[Γενετικοί Αλγόριθμοι] \hfill \\ Οι Γενετικοί αλγόριθμοι ανήκουν στο κλάδο της επιστήμης υπολογιστών και αποτελούν μια μέθοδο αναζήτησης βέλτιστων λύσεων σε προβλήματα βελτιστοποίησης. Είναι χρήσιμοι σε περιπτώσεις όπου ο χώρος αναζήτησης λύσης είναι πολύ μεγάλος και δεν υπάρχει αναλυτική μέθοδος που να μπορεί να βρει το βέλτιστο συνδυασμό τιμών των μεταβλητών του προβλήματος ώστε το υπό εξέταση σύστημα να αντιδρά με βέλτιστο τρόπο. Ο τρόπος λειτουργίας των Γενετικών Αλγορίθμων είναι εμπνευσμένος από τη βιολογία. Χρησιμοποιεί δηλαδή την ιδέα της εξέλιξης μέσω γενετικής μετάλλαξης, φυσικής επιλογής και διασταύρωσης. Για να αξιοποιήσουμε αυτές τις ιδέες, κωδικοποιούμε κάθε πιθανή λύση του προβλήματος σαν ένα συγκεκριμένο γονιδίωμα και ξεκινάμε από έναν τυχαίο πληθυσμό τέτοιων λύσεων/γονιδιωμάτων. Έπειτα, ορίζοντας μια συνάρτηση ικανότητας (\en{fittness function}) που περιγράφει την ποιότητα της λύσης είμαστε σε θέση να αφήσουμε τον μηχανισμό εξέλιξης να δράσει για ορισμένες γενιές ώστε τελικά να έχουν απομείνει και πολλαπλασιαστεί γονιδιώματα που περιγράφουν (σχεδόν) βέλτιστες λύσεις. Οι γενετικοί αλγόριθμοι δεν εγγυώνται την εύρεση της βέλτιστης λύσης.
    
    \hypertarget{_capsule_networks}{
    \item[Νευρωνικά Δίκτυα με Κάψουλες (\en{Capsule Networks})]} \hfill \\ Πρόκειται για βαθιά νευρωνικά δίκτυα που επιδιώκουν να πραγματοποιήσουν ανάστροφα γραφικά για να λύσουν κυρίως προβλήματα αναγνώρισης αντικειμένων σε εικόνες. Αποτελούνται από επίπεδα από κάψουλες. Κάθε κάψουλα είναι σαν μια συνάρτηση η οποία προσπαθεί να προβλέψει τις παραμέτρους στιγμιοτύπου (π.χ. προσανατολισμός, θέση κ.τ.λ.) ενός συγκεκριμένου αντικειμένου και την πιθανότητα ύπαρξής του σε μια περιοχή της εικόνας (δηλαδή στο πεδίο υποδοχής της κάψουλας).
    
    \item[Γραφικά Υπολογιστή] \hfill \\ Αφορά τον κλάδο της επιστήμης υπολογιστών που μελετά μεθόδους για ψηφιακή σύνθεση και χειρισμό οπτικού περιεχομένου. Εμπεριέχει μια δόση τέχνης αφού σχετίζεται με τον σχεδιασμό του περιεχομένου αυτού.
    
    \item[Απόδοση Εικόνας (\en{Rendering})] Είναι η διεργασία δημιουργίας εικόνας από ένα μοντέλο δύο ή τριών διαστάσεων με τη χρήση ενός προγράμματος υπολογιστή. Πολλά μοντέλα ορίζονται σε ένα αρχείο σκηνής (\en{scene file}) το οποίο περιγράφει όλη την πληροφορία της οπτικής σκηνής που θα παραχθεί με την απόδοση εικόνας. Συνήθως, το αρχείο σκηνής περιέχει πληροφορία για τη γεωμετρία, την οπτική γωνία, την υφή, τον φωτισμό και τη σκίαση των αντικειμένων.
    
    \item[Ανάστροφα Γραφικά] \hfill \\ Πρόκειται για την ανάστροφη διαδικασία της απόδοσης εικόνας. Δηλαδή, δοθείσης μιας οπτικής εικόνας, να προσδιοριστεί το αρχείο σκηνής από το οποίο δημιουργήθηκε.
    
    \item[Ακολουθιακά Δεδομένα (\en{Sequential Data})] \hfill \\ Ο όρος αφορά δεδομένα των οποίων τα επιμέρους στοιχεία διατάσσονται σε μια συγκεκριμένη σειρά. Για παράδειγμα, οι λέξεις στον φυσικό λόγο αποτελούν ακολουθιακά δεδομένα. Άλλα παραδείγματα είναι οι ακολουθίες DNA και η τιμή μιας μετοχής στο χρηματιστήριο, όπως αυτή μεταβλαλλεται στον χρόνο.
    \item[Κανονικοποίηση Επιπέδου (\en{Layer Normalization})] \hfill \\ Πρόκειται για μια τεχνική που χρησιμοποιείται για την κανονικοποίηση της κατανομής των τιμών ενεργοποίησης σε κάθε παράδειγμα εισόδου ξεχωριστά \cite{ba2016layer_normalization}. Η τεχνική αυτή μειώνει σημαντικά τον χρόνο εκπαίδευσης (οι συναρτήσεις ενεργοποίησης λειτουργούν στη γραμμική περιοχή τους, γύρω από το μηδέν).
    \item[Ανταγωνιστική Μάθηση (\en{Competitive Learning})] \hfill \\ Αφορά την διαδικασία μη\textendash επιβλεπόμενης μάθησης κατά την οποία διαφορετικοί νευρώνες (ή γενικότερα, υπολογιστικές μονάδες) ανταγωνίζονται για το ποιός θα αναλάβει να \textquote{εξηγήσει} και να μάθει να αναπαριστά την εκάστοτε είσοδο $x_i$ (ενός συνόλου εδομένων). Από την στιγμή που όλοι οι νευρώνες, καθώς το δίκτυο τροφοδοτείται με παραδείγματα, μαθαίνουν να αναπαριστούν καλύτερα τις εισόδους που είναι ήδη καλοί στο να αναπαριστούν, εξειδικεύονται στο να εξηγούν συγκεκριμένα μοτίβα εισόδων ο καθένας. Μια από τις πιο απλές μορφές της ανταγωνιστικής μάθησης είναι η λεγόμενη \textquote{ο νικητής τα παίρνει όλα} (\en{winner takes it all}), όπως παρουσιάζεται στην ενότητα \ref{sec:_SOM} \cite{sammut2011encyclopedia}.
    \item[Αυτοκωδικοποιητής (\en{Autoencoder})] Πρόκειται για μια αρχιτεκτονική τεχνητών νευρωνικών δικτύων που χρησιμοποιείται για την εκμάθηση αποδοτικών (διανυσματικών) αναπαραστάσεων από μη\textendash σεσημασμένα δεδομένα. Σε αυτό το είδος του νευρωνικού δικτύου, δίνεται σαν είσοδος ένα παράδειγμα (διάνυσμα χαρακτηριστικών) και απαιτούμε να λάβουμε στην έξοδο το ίδιο διάνυσμα (μέσω μιας συνάρτησης σφάλματος η οποία συγκρίνει την έξοδο με την είσοδο). Με λίγα λόγια, εκπαιδεύουμε το δίκτυο ώστε να έχει συμπεριφορά ταυτοτικής συνάρτησης. Ο περιορισμός όμως είναι ότι ανάμεσα στο επίπεδο εισόδου και το επίπεδο εξώδου επιβάλουμε (μέσω αρχιτεκτονικής) μια στένωση (bottleneck) με το να χρησιμοποιούμε λιγότερους κόμβους κρυφών επιπέδων από τους κόμβους εισόδου και εξόδου. Έτσι, το δίκτυο μαθαίνει συμπικνωμένες αναπαραστάσεις των δεδομένων. 
 \end{description}