\chapter{Ορισμοί Εννοιών}

Το παρόν παράρτημα περιέχει ορισμούς εννοιών που εισάγονται κατά τη διάρκεια της παρούσας εργασίας. Κατά αυτόν τον τρόπο, δε διακόπτεται η ροή του κυρίως κειμένου.

\begin{description}
    \item[Τεχνητή Νοημοσύνη] \hfill \\ 
    Έχουν υπάρξει πολλοί διαφορετικοί ορισμοί της Τεχνητής Νοημοσύνης: Μερικοί την περιγράφουν σαν εσωτερική διαδικασία της σκέψης που προσομοιάζει αυτή του ανθρώπου ενώ άλλοι ως εξωτερική διαδικασία μαθηματικά βέλτιστης συμπεριφοράς. Σύμφωνα με το κυρίαρχο μοντέλο, η Τεχνητή Νοημοσύνη ασχολείται κυρίως με τη λογική δράση. Ένας ιδανικός ευφυής πράκτορας δρα βέλτιστα σε κάθε περίσταση. Έτσι λοιπόν, η μελέτη της δημιουργίας ευφυών πρακτόρων μπορεί να τεθεί ως ορισμός της Τεχνητής Νοημοσύνης.
    \item[Μηχανική Μάθηση] \hfill \\ A much longer description.   \ldots
    \item[Τεχνητά Νευρωνικά Δίκτυα] \hfill \\ A much longer description.   \ldots
    \item[Βαθιά Μάθηση] \hfill \\ A short one-line description.
    \item[Γνωστική Νευρoεπιστήμη] \hfill \\ A short one-line description. 
    \item[Ταξινόμηση Προτύπων] \hfill \\ A short one-line description.  
    \item[Γραμμικά Διαχωρίσιμες Κλάσσεις] \hfill \\ A short one-line description.
    \item[Γραμμικά Μοντέλα] \hfill \\ A short one-line description.   
    \item[Γενετικοί Αλγόριθμοι] \hfill \\ A short one-line description.
    \item[Επιβλεπόμενη Μάθηση] \hfill \\ A short one-line description.  
    \item[Μη-επιβλεπόμενη Μάθηση] \hfill \\ A short one-line description.
    \item[Ενισχυτική Μάθηση] \hfill \\ A short one-line description.    
 \end{description}