\chapter{Απόδοση Ξενόγλωσσων Όρων}
\begin{center}
\begin{tabular}{ll}
    \large{\textbf{\underline{Ξενόγλωσσος όρος}}} & \large{\textbf{\underline{Ελληνική απόδοση}}}\\
    
    \en{batch learning} & μάθηση κατά δέσμες\\
    \en{online learning} & μάθηση σε ζωντανό χρόνο\\
    \en{supervised learning} & επιβλεπόμενη μάθηση\\
    \en{unpervised learning} & μη-επιβλεπόμενη μάθηση\\
    \en{reinforcement learning} & ενισχυτική μάθηση\\
    \en{capsule networks} & νευρωνικά δίκτυα με κάψουλες\\
    \en{instance based} & βασισμένο σε παραδείγματα\\
    \en{perturbation test} & πείραμα διαταραχής\\
    \en{learning rate} & ρυθμός μάθησης\\
    \en{optimizer} & βελτιστοποιητής\\
    \en{multihead attention} & πολυκέφαλη προσοχή\\
    \en{equivariant} & ισομεταβλητό \\
    \en{invariant} & ανεξάρτητο \\
    \en{recurrent neural network} & επαναλαμβανόμενο νευρωνικό δίκτυο\\
    \en{reconstructor} & ανακατασκευαστής\\
    \en{inverse graphics} & ανάστροφα γραφικά\\
    \en{primary capsules} & κάψουλες πρώτου επιπέδου\\
    \en{digit capsules} & κάψουλες τελευταίου επιπέδου\\
    \en{softmax} & συνάρτηση ομαλής μεγιστοποίησης\\
    \en{squashing function} & συνάρτηση σύνθλιψης\\
    \en{single parent assumption} & υπόθεση μοναδικού πατέρα\\
    \en{similarity score} & σκορ ομοιότητας\\
    \en{coupling coefficients} & βάρη δρομολόγησης \\



\end{tabular}
\end{center}

