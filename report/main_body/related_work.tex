\chapter{Βιβλιογραφική Επισκόπηση}
\label{chap:related_work}

Πριν την έναρξη της εκπόνησης του πρακτικού τμήματος της παρούσας διπλωματικής πραγματοποιήθηκε βιβλιογραφική επισκόπηση προκειμένου να αναζητηθούν εργασίες σχετικές με το θέμα των νευρωνικών δικτύων με κάψουλες. Στο κεφάλαιο αυτό, θα γίνει αναφορά στις σημαντικότερες από αυτές οι οποίες λήφθηκαν υπόψην και ενέμπνευσαν τις μεθόδους που θα αναλύσουμε στο επόμενο κεφάλαιο.\par

Αρχικά, θα παρουσιάσουμε τις τρείς βασικές δημοσιεύσεις των \en{Hinton G. et al.} που θεμελίωσαν τη θεωρία πίσω από τα νευρωνικά δίκτυα με κάψουλες σε ένα πλαίσιο επιβλεπόμενης μάθησης. Έπειτα, θα αναφερθούμε στις ποικίλες παραλλαγές αυτών, όπως προκύπτουν από την τροποποίηση της αρχιτεκτονικής ή του αλγορίθμου δρομολόγησης. Στη συνέχεια, θα γίνει λόγος για τα νευρωνικά δίκτυα με κάψουλες σε περιβάλλον μη\textendash επιβλεπόμενης μάθησης. Τέλος, θα αναλυθούν συνοπτικά εργασίες οι οποίες επιλύουν αποτελεσματικά το γενικκότερο πρόβλημα της γενίκευσης σε νέες οπτικές γωνίες χρησιμοποιώντας αρχιτεκτονικές διαφορετικές από αυτή των νευρωνικών δικτύων με κάψουλες.\par

\section{Θεμελίωση Θεωρίας Νευρωνικών Δικτύων με Κάψουλες}

Όπως έχουμε αναφέρει, η ιδέα των νευρωνικών δικτύων με κάψουλες δεν είναι καινούρια αφού παρουσιάστηκε για πρώτη φορά από τους \en{Hinton G. et al.} το 2011. Παρόλα αυτά, σχετικά πρόσφατα, μετά από διαδοχικές δημοσιεύσεις, ωρίμασε και πέτυχε αξιοσημείωτα αποτελέσματα σε σύνολα δεδομένων όπως το \en{MultiMNIST}\cite{sabour2017dynamic}. Στην ενότητα αυτή θα κάνουμε λόγο για τα πρώτα τρία βασικά έργα πάνω στην εν λόγω αρχιτεκτονική τεχνητών νευρωνικών δικτύων. Πιο αναλυτικά, θα ξεκινήσουμε από τη δημοσίευση στην οποία πρωτοπαρουσιάστικε η ιδέα και θα καταλήξουμε στην πιο σύνθετη έκδοση των νευρωνικών δικτύων με κάψουλες για επιβλεπόμενη μάθηση που με τις επιδόσεις της στο σύνολο δεδομένων \en{smallNORB}\cite{lecun2004learning} κέντρισε το ενδιαφέρον των ερευνητών. 

\subsubsection{\en{Transforming Autoencoders}}

Στο έργο των \en{Hinton G. et al.}\footnote{Σε ελεύθερη μετάφραση: \textquote{Αυτο\textendash κωδικοποιητές Μετασχηματισμού}.} \cite{hinton2011transforming} παρουσιάζεται για πρώτη φορά η ιδέα των νευρωνικών δικτύων με κάψουλες. Η ιδέα απορρέει από την παρατήρηση ότι οι παρούσες μέθοδοι αναγνώρισης αντικειμένων σε εικόνες είναι ανεπαρκείς (για τους λόγους που αναφέραμε στην ενότητα \ref{sec:capsule_theory}). Έτσι, προκειμένου να γίνεται αποδοτικότερη αναγνώριση αντικειμένων (σε νέες οπτικές γωνίες), προτείνεται η αρχιτεκτονική του \textquote{αυτο\textendash κωδικοποιητή μετατροπέα} (\en{transforming auto\textendash encoder}). Η αρχιτεκτονική αυτή αποτελείται από ένα επίπεδο από \textquote{κάψουλες}, όπως φαίνεται στο σχήμα ...\footnote{Στην πρώιμη μορφή τους, οι κάψουλες διέφεραν από τη γενική μορφή που περιγράψαμε στην προηγούμενη ενότητα.}.\par

Κάθε κάψουλα αποτελείται από τις \textquote{μονάδες αναγνώρισης} (\en{recognition units}) οι οποίες παράγουν τις παραμέτρους στιγμιοτύπου καθώς και μια τιμή που συμβολίζει την πιθανότητα η οντότητα που αναγνωρίζει η κάψουλα να είναι παρούσα στο οπτικό της πεδίο (στο τμήμα της εικόνας εισόδου με το οποίο συνδέονται οι μονάδες αναγνώρισής της \textemdash στην περίπτωσή μας, σε ολόκληρη την εικόνα). Οι μονάδες παραγωγής είναι υπεύθυνες και για τον μετασχηματισμό από τον χώρο εικονοστοιχείων της εικόνας εισόδου σε έναν χώρο όπου οι μετασχηματισμοί οπτικής γωνίας (μετατόπισή, περιστροφή κ.α.) είναι γραμμικοί. Στη συνέχεια, κάθε κάψουλα τροφοδοτείται με τους καθολικούς (\en{global}) μετασχηματισμούς που συνδέουν την εικόνα εισόδου με την εικόνα εξόδου οι οποίοι εφαρμόζονται στις υπολογισμένες παραμέτρους. Έτσι, οι παράμετροι στιγμιοτύπου της κάθε κάψουλας πλέον εκφράζουν τις παραμέτρους στιγμιοτύπου του αντικειμένου εξόδου. Τέλος, η εικόνα ανακατασκευάζεται από τις μονάδες παραγωγής \en{generation units} τις οποίες κάθε κάψουλα διαθέτει. Αυτές ουσιαστικά διαβάζουν τις (μετασχηματισμένες) παραμέτρους στιγμιοτύπου και συνεισφέρουν στην ανακατασκευή της εικόνας εξόδου. Θεωρητικά, κάθε κάψουλα αναγνωρίζει ένα συγκεκριμένο τμήμα του αντικειμένου της εικόνας και όλες μαζί οι κάψουλες, συνθέτουν τα τμήματα που αναπαριστούν σιωπηρά (\en{implicitly}). Φυσικά, αν η τιμή πιθανότητας για μια κάψουλα είναι κοντά στο μηδέν, η συνεισφορά της θα είναι αμελητέα.\par

Ουσιαστικά, η αρχιτεκτονική του αυτο\textendash κωδικοποιητή που παρουσιάζεται πραγματοποιεί έναν μετασχηματισμό από τον χώρο των εικονοστοιχείων σε έναν χώρο αναπαράστασης όπου οι γεωμετρικοί μετασχηματισμοί περιγράφονται με γραμμικές σχέσεις. Στον χώρο αυτό εφαρμόζεται ένας γραμμικός μετασχηματισμός στις παραμέτρους της κάθε κάψουλας και έπειτα, οι παράμετροι αποκωδικοποιούνται πίσω στον χώρο των εικονοστοιχείων όπου και λαμβάνεται η μετασχηματισμένη εικόνα.\par

Στη δημοσίευση γίνονται πειράματα κυρίως στο σύνολο δεδομένων \en{MNIST}\cite{lecun1998gradientMNIST} για μικρές μετατοπίσεις των ψηφίων κατά τον $x$ και $y$ άξονα. Από αυτά τα πειράματα φαίνεται ότι οι παράμετροι σωστά εντοπίζουν τη θέση των αντικειμένων αλλά το οπτικό πεδίο της κάθε κάψουλας, μετά την εκπαίδευση, δεν είναι τοπικά προσδιορισμένο. Με άλλα λόγια, χρησιμοποιείται το σύνολο της εικόνας από τις μονάδες αναγνώρισης της κάθε κάψουλας για την εξαγωγή των παραμέτρων της. Επιπρόσθετα πειράματα έγιναν χρησιμοποιώντας το σύνολο δεδομένων \en{smallNORB}\cite{lecun2004learning} προκειμένου να διερευνηθεί η επίδοση της αρχιτεκτονικής σε σύνθετες μεταβολές της οπτικής γωνίας (\en{3-D Orientation}) που αναπαριστώνται με πίνακες $3\times3$. Για αυτό το σύνολο αυξήθηκε ο αριθμός των καψουλών του δικτύου και αυτές είχαν πεδίο υποδοχής που δεν κάλυπτε όλη την εικόνα. Όπως φαίνεται και στη δημοσίευση\cite{hinton2011transforming}, οι παραχθείσες εικόνες φαίνονται θολές.\par

Αν και το έργο που περιγράφουμε έχει ιδιαίτερη αξία αφού θεμελίωσε τις αρχές των νευρωνικών δικτύων με κάψουλες και διατύπωσε ορισμένα προβλήματά τους (π.χ. \en{crowding}), δεν μπορούσε να έχει πρακτική εφαρμογή λόγω της επίδοσής του στα σύνολα δεδομένων που δοκιμάστηκε. Επιπλέον, το γεγονός ότι για την εκπαίδευσή του, εκτός από τις εικόνες εισόδου και εξόδου έπρεπε να παρέχεται και η σχέση μεταξύ αυτών ήταν ένας ακόμη ανασταλτικός παράγοντας. Επιπρόσθετα, δεν παρείχε κάποιον ρητό τρόπο για την ανάθεση μερών του αντικειμένου σε αυτό. Όλα αυτά οδήγησαν σε απόπειρες βελτίωσης της αρχιτεκτονικής από το επόμενο έργο που θα παρουσιάσουμε. 

\subsubsection{\en{Dynamic Routing Between Capsules}}

Το έργο των \en{Sabour S. et al.}\footnote{Σε ελεύθερη μετάφραση: \textquote{Δυναμική Δρομολόγηση μεταύ Καψουλών}.} \cite{sabour2017dynamic} εξελίσσει την προηγούμενη μελέτη των νευρωνικών δικτύων με κάψουλες προτείνοντας έναν αλγόριθμο δρομολόγησης μέσω συμφωνίας. Με αυτόν, καθίσταται δυνατή η σύνθεση αντικειμένων από τα επιμέρους τμήματά του. Επιπλέον, αναθεωρεί τη δομή της κάψουλας η οποία πλέον είναι απόλυτα σύμφωνη με τον ορισμό που δώσαμε στην ενότητα \ref{sec:capsule_theory}. Δηλαδή, οι κάψουλες πλέον δεν αποτελούνται από δύο διαφορετικές ομάδες από τεχνητούς νευρώνες αλλά είναι ομάδες νευρώνων και η κάθε μια αναπαριστά ιδιότητες της συγκεκριμένης οντότητας που αναγνωρίζει. Οι ιδιότητες της αναγνωρισμένης οντότητας αναπαριστώνται με ένα διάνυσμα ενώ η βεβαιότητα αναγνώρισής της στην εικόνα εισόδου (η τιμή πιθανότητας) κωδικοποιείται στο μήκος του διανύσματος αυτού. Οι ανωτέρω βελτιώσεις, σε συνδυασμό με μια καινούρια αρχιτεκτονική είχαν σαν αποτέλεσμα βελτιωμένες (για την εποχή) επιδόσεις στο \en{MNIST}\cite{lecun1998gradientMNIST} και στο \en{MultiMNIST}\cite{sabour2017dynamic}.\par

Αναλυτικότερα για τη δομή του δικτύου από κάψουλες, αυτή αποτελείται από τρία επίπεδα τεχνητών νευρώνων. Το πρώτο είναι ένα κλασσικό συνελικτικό επίπεδο, όπως περιγράφηκε στην ενότητα \ref{sec:_cnn}. Το δεύτερο επίπεδο είναι και αυτό συνελικτικό και μαζί με το πρώτο αναλαμβάνουν τον ρόλο του μετασχηματισμού του χώρου των εικονοστοιχείων (χώρος εισόδου) σε έναν χώρο όπου οι νευρικές αποκρίσεις μεταβάλλονται γραμμικά καθώς αλλάζει η γωνία θέασης της εικόνας εισόδου. Στη συνέχεια, οι χάρτες χαρακτηριστικών (οι νευρικές αποκρίσεις) του δεύτερου συνελικτικού επιπέδου ομαδοποιούνται σε διανύσματα τα οποία και αποτελούν τις παραμέτρους στιγμιοτύπου του πρώτου επιπέδου από κάψουλες (\en{PrimaryCaps}). Μέσω της εκπαίδευσης, οι νευρώνες που προηγούνται των διανυσμάτων της κάθε κάψουλας δυνητικά μαθαίνουν να συσχετίζουν τις τιμές μεταξύ τους με τέτοιο τρόπο ώστε να αναπαριστούν ιδιότητες του ίδιου τμήματος αντικειμένου. Το τελευταίο επίπεδο είναι ένα επίπεδο από κάψουλες το οποίο αποτελεί και το επίπεδο εξόδου (ονομάζεται ως \en{DigitCaps}). Ο αριθμός των καψουλών στο επίπεδο εξόδου είναι τόσος όσος και ο αριθμός των κλάσεων ταξινόμησης\footnote{Σε μερικές εξαιρέσεις, χρησιμοποιείται μια παραπάνω κάψουλα εξόδου για την περίπτωση όπου το αντικείμενο εξόδου δεν ανήκει σε καμία από τις κλάσεις για τις οποίες το δίκτυο έχει εκπαιδευτεί να αναγνωρίζει.}. Τέλος, προαιρετικά προτείνεται η χρήση ενός αποκωδικοποιητή για την ανακατασκευή της αρχικής εικόνας με είσοδο το διάνυσμα της κάψουλας που εμπεριέχει το διάνυσμα ιδιοτήτων του αντικειμένου με το μεγαλύτερο μήκος (ονομάζεται διάνυσμα πρόβλεψης).\par

Για τη διαμόρφωση των διανυσμάτων του τελευταίου επιπέδου από κάψουλες (\en{DigitCaps}) χρησιμοποιείται ένας αλγόριθμος δρομολόγησης με συμφωνία του οποίου η βασική λειτουργία περιγράφηκε στην ενότητα \ref{sec:capsule_theory}. Συγκεκριμένα, με τον προτεινόμενο αλγόριθμο \textquote{Δυναμικής Δρομολόγησης μέσω Συμφωνίας} (\en{Dynamic Routing by Agreement}), οι κάψουλες του προηγούμενου επιπέδου παράγουν μια πρόβλεψη για τις παραμέτρους στιγμιοτύπου της κάθε κάψουλας του επόμενου επιπέδου. Τις προβλέψεις αυτές τις δρομολογούν στο επόμενο επίπεδο βεβαρημένες από τις \textquote{παραμέτρους σύζευξης} που προσαρμόζονται από τον εν λόγο αλγόριθμο. Όταν πολλές προβλέψεις συμφωνούν για τις παραμέτρους στιγμιοτύπου μιας κάψουλας, τότε με αυτόν τον τρόπο συνδιαμορφώνουν τις παραμέτρους της και αυτή αποκτά μεγάλη τιμή πιθανότητας (το διάνυσμά της έχει μεγάλο μέτρο). Μια ακόμα βελτίωση που οφείλεται στη χρήση αλγορίθμου δρομολόγησης είναι ότι σε αντίθεση με την προηγούμενη μέθοδο, πλέον δεν απαιτείται να παρέχεται κατά την εκπαίδευση κάποιος πίνακας μετασχηματισμού. Αντίθετα, το δίκτυο αποθηκεύει εσωτερικά πίνακες μετασχηματισμού οι οποίοι μαθαίνουν (μέσω εκπαίδευσης) να αναπαριστούν τις (ανεξάρτητες\textendash στιγμιοτύπου) σχέσεις τμημάτων - όλου.\par

Η δημοσίευση \en{Dynamic Routing Between Capsules} παρείχε υποσχόμενα πειραματικά αποτελέσματα. Πιο συγκεκριμένα, δοκιμάστικε στο σύνολο δεδομένων \en{MNIST}\cite{lecun1998gradientMNIST} όπου και είχε 0.25\% σφάλμα ελέγχου (\en{test error}) με μόλις 8.2\en{M} παραμέτρους. Για σύγκριση, ένα τυπικό \en{baseline} συνελικτικό δίκτυο πετυχαίνει 0.39\% σφάλμα ελέγχου (\en{test error}) με πολύ περισσότερες παραμέτρους (35.4\en{M}). Αξιοσημείωτες επιδόσεις παρατηρήθηκαν και στο \en{MultiMNIST}\cite{sabour2017dynamic} σύνολο δεδομένων το οποίο αποτελείται από αριθμούς με υψηλή επικάλυψη μεταξύ τους. Σε αυτό το σφάλμα ελέγχου ήταν 5.2\%, πολύ μικρότερο από αυτό του συνελικτικού μοντέλου (8.1\%). Επίσης, βέλτιστα (για την εποχή) αποτελέσματα παρατηρήθηκαν στα σύνολα δεδομένων \en{affNIST} και \en{smallNORB}. Ειδικά οι υψηλές επιδώσεις στο πρώτο σύνολο, όπου περιέχει ψηφία μετασχηματισμένα από διάφορους αφινικούς μετασχηματισμούς, αποδεικνύει την ευρωστία των δικτύων με κάψουλες σε μεταβολές της οπτικής γωνίας. Τέλος, το δίκτυο δοκιμάστηκε στο (σύνθετο) σύνολο δεδομένων CIFAR10 αλλά η επίδοσή του σε αυτό δεν ήταν εντυπωσιακή (10.6\% σφάλμα ελέγχου).

\subsubsection{\en{Matrix Capsules with EM Routing}}

Η επιστημονική μελέτη των \en{Hinton G. et al.}\footnote{Σε ελεύθερη μετάφραση: \textquote{Πινακοειδής Κάψουλες με Αλγόριθμο Δρομολόγησης Μεγιστοποίησης Προσδοκιών}.}\cite{hinton2018matrix} βελτιώνει την προηγούμενη υλοποίηση τροποποιώντας την αρχιτεκτονική του νευρωνικού δικτύου από κάψουλες (αυξάνοντας τον συνολικό αριθμό παραμέτρων) και προτείνοντας έναν νέο αλγόριθμο δρομολόγησης μέσω συμφωνία βασιζόμενο στον αλγόριθμο Μεγιστοποίησης Προσδοκιών (\en{Expectation Maximization}). \par

Πιο αναλυτικά, οι βασικότερες τροποποιήσεις της προηγούμενης μελέτης είναι οι εξής:
\begin{enumerate}
    \item Η κάθε κάψουλα διαθέτει ξεχωριστή λογιστική μονάδα (\en{logistic unit}) για την αναπαράσταση της πιθανότητας ύπαρξης της οντότητας που αναγνωρίζει. Αυτός ο τρόπος, σύμφωνα με τους \en{Hinton G. et al.}\cite{hinton2018matrix} είναι καλύτερος από τη κωδικοποίηση της τιμής πιθανότητας στο μήκος του διανύσματος παραμέτρων στιγμιοτύπου.
    \item Σαν μετρική ομοιότητάς μεταξύ των ψήφων χρησιμοποιείται ο αρνητικός λογάριθμος της διακύμανσης (\en{variance}) των Γκαυσσιανών συστάδων. Αυτή η μετρική ομοιότητας είναι καλύτερη από την ομοιότητα συνημιτόνου (\en{cosine similarity}) καθώς είναι πιο ευαίσθητη στη περιοχή υψηλής ομοιότητας\footnote{Με άλλα λόγια, μπορεί καλύτερα να διακρίνει μια σχετικά καλή ομοιότητα από μια άριστη ομοιότητα.}.
    \item Στην νέα μελέτη προτείνεται μια ελαφρώς τροποποιημένη δομή κάψουλας η οποία ενθυλακώνει τις παραμέτρους στιγμιοτύπου υπο τη μορφή πίνακα πόζας με $n$ στοιχεία. Αυτή η αλλαγή επιτρέπει στους πίνακες μετασχηματισμού να έχουν μέγεθος $n^2$ και όχι μόνο $n$.
    \item Εισάγεται μια νέα πολυεπίπεδη αρχιτεκτονική (βλ. σχήμα ....) η οποία περιλαμβάνει συνελικτικά επίπεδα από κάψουλες προκειμένου να διαμοιράζεται η γνώση (που αποθηκεύεται στη μορφή πινάκων μετασχηματισμού) στον χώρο.
\end{enumerate}\par

Με τα πειράματα που έγιναν στο προτεινόμενο μοντέλο μηχανικής μάθησης αποδεικνύεται η αποδοτικότερη αναγνώριση αντικειμένων όταν αυτά αναπαρίστανται σε εικόνες με διαφορετικές γωνίες λήψης. Για παράδειγμα, για το σύνολο δεδομένων \en{smallNORB} επιτυγχάνεται σφάλμα ελέγχου ίσο με 1.4\% (πολύ μικρότερο σε σχέση με το σφάλμα 5.2\% του βασικού μοντέλου - αποτελούμενου από συνελικτικά επίπεδα). Επιπλέον, υψηλές επιδόσεις παρατηρήθηκαν όταν δοκιμάστικε η προτεινόμενη αρχιτεκτονική νευρωνικού δικτύου με κάψουλες στο ίδιο σύνολο δεδομένων αλλά σε οπτικές γωνίες απεικονιζόμενων αντικειμένων που δεν είχε εκπαιδευτεί (\en{novel viewpoints}). Τέλος, το μοντέλο φάνηκε να είναι εύρωστο σε επιθέσεις τύπου λευκού\textendash κουτιού (\en{white\textendash box adversarial attacks})\cite{goodfellow2014explaining}\footnote{Έχει δειχθεί ότι δεν ισχύει το ίδιο για επιθέσεις τύπου μαύρου\textendash κουτιού (\en{black\textendash box adversarial attacks})}. 

\section{Παραλλαγές Νευρωνικών Δικτύων με Κάψουλες}

Στην ενότητα αυτή θα γίνει σύντομη αναφορά στις βασικότερες έρευνες που σχετίζονται άμεσα με τα νευρωνικά δίκτυα από κάψουλες σε περιβάλλον επιβλεπόμενης μάθησης. Οι έρευνες αυτές κυρίως εστιάζουν σε τροποποιήσεις του αλγορίθμου δρομολόγησης και της αρχιτεκτονικής του δικτύου. Ακόμα, περιλαμβάνονται ορισμένες εργασίες που πειραματίζονται εκτενώς με τις βασικές υλοποιήσεις, όπως τις περιγράψαμε παραπάνω.\par

Κατά τη διάρκεια της βιβλιογραφικής μελέτης των νευρωνικών δικτύων με κάψουλες απαιτείται να έχουμε υπ'όψη τα εξής κριτήρια:
\begin{itemize}
    \item Αν οι βασικές ιδιότητες που σχετίζονται με την αποδοτική διαχείριση των αντικειμένων υπό διαφορετικές οπτικές γωνίες διατηρούνται (π.χ. εύρωστες εσωτερικές αναπαραστάσεις που μεταβάλλονται ανάλογα με τις αλλαγές στην οπτική γωνία, δυνατότητα αποθήκευσης γνώσης ανεξάρτητη από τη γωνία θέασης κ.α.).
    \item Αν υπάρχουν αλλαγές στις υποθέσεις που αφορούν τις σχέσεις μέρους\textendashόλου.
    \item Αν οι κάψουλες ενεργοποιούνται μέσω πολυδιάστατης σύμπτωσης \en{high\textendash dimensional coincidences}
    \item Πώς διαχειρίζεται το προτεινόμενο σύστημα την εγγενή αβεβαιότητά της σύνθεσης ενός αντικειμένου από τα τμήματά του. \cite{de2020introducing}
\end{itemize}\par

Σημειώνουμε ότι στις βιβλιογραφικές μελέτες στις οποίες αναφερόμαστε παρακάτω αποφύγαμε να συμπεριλάβουμε τα έργα που παρουσιάζουν μεγάλες αποκλίσεις από τα βασικά κριτήρια των νευρωνικών δικτύων με κάψουλες.\par

\subsubsection{\en{Capsule Routing via Variational Bayes}}

Η εν λόγω μελέτη\footnote{Σε ελεύθερη μετάφραση: \textquote{Δρομολόγηση Καψουλών με Μπεϋζιανή Διακύμανση}.}\cite{De_Sousa_Ribeiro_Leontidis_Kollias_2020_Capsule_Routing} βασίζεται στην \cite{hinton2018matrix} και την βελτιώνει προτείνοντας έναν διαορετικό αλγόριθμο δρομολόγησης μέσω συμφωνίας. Πιο συγκεκριμένα, με τον αλγόριθμο δρομολόγησης βασισμένο στη συμπερασματολογία διακύμανσης (\en{Variational Inference}) - ονομάζεται δρομολόγηση μπεϋζιανής διακύμανσης (\en{Variational Bayed Routing}) είναι εφικτή η μοντελοποίηση αβεβαιότητας στις παραμέτρους της κάψουλας (εκτός από τους συντελεστές δρομολόγησης). Με αυτήν την πιθανοκρατική προσέγγιση, είναι εφικτή η τροποποίηση των πρότερων πιθανοτήτων της κάθε κάψουλας για καλύτερο έλεγχο της πολυπλοκότητάς τους και για αποφυγή του προβλήματος της κατάρρευσης διασποράς (\en{variance collapse}). Επιπλέον, δείχνουν τον τρόπο με τον οποίοι ένα νευρωνικό δίκτυο από κάψουλες μπορεί να μετατραπεί σε αυτο\textendash κωδικοποιητή διακύμανσης (\en{variational auto\textendash encoder}). Τέλος, παρέχουν μερικές οδηγίες για την εκπαίδευση του προτεινόμενου μοντέλου (αρχικοποίηση βαρών και σχέδια κανονικοποίησης). \par

Οι πειραματισμοί του προτεινόμενου μοντέλου στα σύνολα δεδομένων \en{smallNORB, SVHN, MNIST} και \en{affNIST} αποδεικνύουν την ισχυριζόμενη βελτίωση της βασικής υλοποίησης των νευρωνικών δικτύων με κάψουλες. Πιο αναλυτικά, στο σύνολο δεδομένων \en{smallNORB}\cite{lecun2004learning} επιτυγχάνεται μείωση του σφάλματος ελέγχου στην τιμή 1.55\% (σε αντίθεση με 1.8\% όπως προκύπτει από το \cite{hinton2018matrix}) χρησιμοποιώντας μόλις τον μισό αριθμό από κάψουλες. Βελτιωμένα αποτελέσματα παρατηρήθηκαν και στα υπόλοιπα σύνολα δεδομένων αλλά και σε πειράματα που δοκιμάζουν την ικανότητα γενίκευσης του δικτύου και την ευρωστία του σε αινικούς μετασχηματισμούς. Τέλος, αποδηκνύεται ότι ένα δίκτυο που χρησιμοποιεί τον προτεινόμενο αλγόριθμο συγκλίνει κατά 20\% γρηγορότερα, με αυξημένη αριθμητική ευσταθής.

\subsubsection{\en{Introducing Routing Uncertainty in Capsule Networks}}
Η επόμενη δημοσίευση που εξετάζουμε \footnote{Σε ελεύθερη μετάφραση: \textquote{Εισάγοντας Αβεβαιότητα Δρομολόγησης στα Νευρωνικά Δίκτυα από Κάψουλες}.}\cite{de2020introducing} τροποποιεί την προηγούμενη υλοποίηση ώστε να είναι πιο αποδοτική με βελτιωμένα πειραματικά αποτελέσματα. Αρχικά, εντοπίζει ορισμένα μειονεκτήματα των τοπικών, επαναλληπτικών αλγορίθμων δρομολόγησης τα οποία είναι:
\begin{itemize}
    \item Το υψηλό υπολογιστικό κόστος ενός επαναληπτικού, αλγορίθμου δρομολόγησης που λαβάνει χώρα μεταξύ δύο διαδοχικών επιπέδων από κάψουλες.
    \item Κατά τη δρομολόγηση της πληροφορίας από το ένα επίπεδο καψουλών στο επόμενο λαμβάνονται υπόψη μόνο τα τοπικά συμφραζόμενα (\en{local context}), δηλαδή η πληροφορία μεταξύ των δύο επιπέδων.
    \item Η τάση για υπερπροσαρμογή ή υποπροσαρμογή (\en{overfitting/underfitting}) ανάλογα με την επιλογή των αριθμών επανάληψης του αλγορίθμου δρομολόγησης (\en{routing iterations}).
\end{itemize}
Για τον σκοπό αυτό, προτείνεται η αντικατάσταση των τοπικών επαναλήψεων (\en{local iterations}) με μια \textquote{σφαιρική εικόνα} (\en{global view}) βασισμένη στην προσέγγιση της εκ των υστέρων πιθανότητας διακύμανσης (\en{variational posterior}) στις συνδέσεις μέρους - όλου σε ένα πιθανοκρατικό μοντέλο. Η χρήση καθολικών κρυφών μεταβλητών (\en{global latent variables}) που επηρεάζουν άμεσα την αντικειμενική συνάρτηση (\en{obective function}) προσδίδει στο  την ικανότητα για εποπτικότερη δρομολόγηση της πληροφορίας. Οι μεταβλητές αυτές ενημερώνονται μεροληπτικά (\en{discriminatively}) σύμφωνα με την αρχή του ελάχιστου μήκους περιγραφής (\en{minimum description length}) της θεωρίας πληροφορίας (\en{information theory}).\par

Τα εκτενή πειράματα στο σύνολο δεδομένων \en{smallNORB} αποδεικνύουν ότι ακόμα και με μικρότερο αριθμό παραμέτρων σε σχέση με τις προηγούμενες υλοποιήσεις των νευρωνικών δικτύων με κάψουλες, η επίδοση του δικτύου είναι ελαφρώς βελτιωμένη. Πολλά πειράματα επίσης διενεργήθηκαν με σκοπό να διασφαλιστεί ότι διατηρούνται οι βασικές ιδιότητες του εν λόγο είδους νευρωνικών δικτύων. Ενδεικτικά, εκτός από τα πειράματα στο σύνολο δεδομένων \en{smallNORB} και \en{MultiMNIST}, έγιναν πειράματα σχετικά με τη δυνατότητα γενίκευσης σε νέες οπτικές γωνίες, την ευρωστία σε αφινικούς μετασχηματισμούς των εικόνων εισόδου για τους οποίους το δίκτυο δεν έχει εκπαιδευτεί αλλά και την ικανότητά του να εκπαιδεύεται αποδοτικά με λίγα παραδείγματα (\en{Few\textendash Shot Learning}). Σε όλα τα πειράματα, οι επιδόσεις ήταν πλήρως ικανοποιητικές, αποδηκνείωντας έτσι ότι τηρούνται οι βασικές υποθέσεις των νευρωνικών δικτύων με κάψουλες.

\subsubsection{\en{Group Equivariant Capsule Networks}}

Το έργο των \en{Lenssen et al.} \footnote{Σε ελεύθερη μετάφραση: \textquote{Νευρωνικά Δίκτυα με Κάψουλες Ομάδας Ισοδύναμης Διακύμανσης}.} \cite{lenssen2018group} προτείνει ένα τροποποιημένο είδος από κάψουλες και έναν αλγόριθμο δρομολόγησης βασιζόμενα στη θεωρεία ομάδων. Προκύπτει, από τον συνδειασμό μελετών τόσο στο αντικείμενο των νευρωνικών δικτύων με κάψουλες όσο και στην μελέτη που εισήγαγε τα δίκτυα συνέλιξης ομάδας\cite{cohen2016group}. Με αυτόν τον τρόπο, το προτεινόμενο μοντέλο αποδεικνήεται ότι εγγυάται τις ιδιότητες της ανεξαρτησίας των παραμέτρων ενεργοποίησης των καψουλών και της ισοδύναμης διακύμανσης των παραμέτρων πόζας (ανάλογα με τις μεταβολές της οπτικής γωνίας του αντικειμένου εισόδου). Ιδιαίτερα ενδιαφέρον είναι ο τρόπος με τον οποίο δημιουργείται το πρώτο επίπεδο από κάψουλες (\en{primary capsules}). Αναλυτικότερα, δεν χρησιμοποιούνται κάποια προσαρμοζόμενα φίλτρα από τον αλγόριθμο εκπαίδευσης αλλά γίνεται χρήση των (στατικών) φίλτρων \en{Sobel}. Τα πειράματα περιορίστηκαν στο σύνολο δεδομένων \en{MNIST} όπου επιτεύχθηκε εκπληκτική ακρίβεια ταξινόμησης των ψηφίων (98.42\%) όταν αυτά είχαν περιστραφεί τυχαία με πολλαπλάσια των $90^{\circ}$ και ενώ το δίκτυο είχε εκπαιδευτεί με ψηφία χωρίς κανένα μετασχηματισμό.


\subsubsection{\en{CapsuleGAN: Generative Adversarial Capsule Network}}

Στην δημοσίευση των \en{Jaiswal et al.} \footnote{Σε ελεύθερη μετάφραση: \textquote{Παραγωγικό Αντιπαραθετικό Δίτκτυο με Κάψουλες}.}\cite{jaiswal2018capsulegan} εφαρμόζεται η αρχιτεκτονική του νευρωνικού δικτύου με κάψουλες, όπως παρουσιάζεται από τους \en{Sabour et al.}\cite{sabour2017dynamic} στο πλαίσιο των παραγωγικών, αντιπαραθετικών δικτύων. Συγκεκριμένα, πρόκειται περισσότερο για μια εφαρμογή που αντικαθίσταται το συνελικτικό δίκτυο διάκρισης (\en{convolutional descriminative network}) ενός παραγωγικού αντιπαραθετικού δικτύου (\en{Generative Adversarial Network}) με ένα δίκτυο διάκρισης από κάψουλες. Για την εκπαίδευση του δικτύου, χρησιμοποιούν μια συνάρτηση κόστους που προκαλεί το παιχνίδι αντιπαράθεσης (\en{adversarial game}) μεταξύ της γεννήτριας και του δικτύου διάκρισης, τροποποιημένη κατάλληλα για ένα δίκτυο διάκρισης από κάψουλες. Μέσα από τα πειράματα στα \en{MNIST} και \en{CIFAR10} σύνολα δεδομένων, προκύπτει ότι ένα Παραγωγικό Αντιπαραθετικό Δίτκτυο με Κάψουλες έχει καλύτερη επίδοση από ένα αντίστοιχο δίκτυο με αμιγώς συνελικτικά επίπεδα. Οι βελτιωμένες επιδόσεις εντοπίστηκαν στη μοντελοποίησης πιθανοτικής κατανομής των δεδομένων εικόνων (όπως προκύπτουν από τη μετρική παραγωγικής αντιπαράθεσης (\en{GAM})\cite{im2016generative} και από τα πειράματα ταξινόμησης εικόνων ημι\textendash επιβλεπόμενης μάθησης).

\subsubsection{\en{MS\textendash CapsNet: A Novel Multi\textendash Scale Capsule Network}}

Στη μελέτη των \en{Xiang et al.} \footnote{Σε ελεύθερη μετάφραση: \textquote{Ένα Νέο Πολυ\textendash Κλιμακωτό Νευρωνικό Δίκτυο με Κάψουλες}.} \cite{xiang2018ms} παρουσιάζεται μια νέα αρχιτεκτονική νευρωνικών δικτύων με κάψουλες που βελτιώνει την ικανότητα αναπαράστασης ιεραρχικής πληροφορίας από τις κάψουλες, μειώνοντας παράλληλα την υπολογιστική πολυπλοκότητα. Η ιδέα πίσω από αυτή την τροποποίηση είναι ότι εξάγγοντας πλουσιότερες αναπαραστάσεις, είναι δυνατή η βελτίωση της επίδοσης σε σύνθετα δεδομένα εισόδου. Επιπλέον, τροποποιείται η \textquote{τεχνική εγκατάλειψης} (\en{dropout}) προκειμένου να μπορεί να εφαρμοστεί σε ένα επίπεδο από κάψουλες. Τέλος, η αρχιτεκτονική δοκιμάζεται σε εργασίες ταξινόμησης στα σύνολα \en{FashionMNIST} και \en{CIFAR10} όπου παρατηρούνται βελτιωμένα αποτελέσματα (ακρίβεια 0.927 και 0.757 αντίστοιχα με λιγότερες από τις μισές παραμέτρους) σε αντιπαραβολή με την υλοποίηση \cite{sabour2017dynamic}.\par

Η αρχιτεκτονική του πρώτου και δεύτερου επιπέδου του δικτύου φαίνεται στο σχήμα ...  . Έπειτα ακολουθεί το τελευταίο επίπεδο από κάψουλες (παρόμοια με το έργο \cite{sabour2017dynamic}). Χωρίς να εμβαθύνουμε ιδιαίτερα, η εξαγωγή χαρακτηριστικών γίνεται πολυκλιμακωτά, με το πρώτο παρακλάδι να παράγει κωδικοποιήσεις σημασιολογικής πληροφορίας (πληροφορίας ανώτερης τάξης), το δεύτερο να παράγει διανύσματα που κωδικοποιούν πληροφορία μέσης τάξης και το τελευταίο, να έχει ως έξοδο τα ακατέργαστα χαρακτηριστικά. Τα εξαγόμενα χαρακτηριστικά οργανώνονται σε κάψουλες που εμπεριέχουν διανύσματα διαστατικότητας 12, 8 και 4 αντίστοιχα (ανάλογα με το παρακλάδι από το οποίο προκύπτουν). Τέλος, μέσω των πινάκων μετασχηματισμών $W, V, U$, παράγονται ψήφοι ίσου μεγέθους  $\hat{u^1_{j|i}}, \hat{u^2_{j|i}}, \hat{u^3_{j|i}}$ για κάθε ζεύγος $i \rightarrow j$ που συνδέονται σηρειακά (\en{concatenate}) σε ένα διάνυσμα $\hat{u_{j|i}} = concat(\hat{u^1_{j|i}}, \hat{u^2_{j|i}}, \hat{u^3_{j|i}})$.\par

Δυστυχώς, πέρα από τα προαναφερθέντα πειράματα, το δίκτυο δεν εξετάζεται κατά πόσον τηρεί τις θεμελιώδεις υποθέσεις των νευρωνικών δικτύων με κάψουλες. Επιπλέον, οι πίνακες $W, V, U$ δεν έχουν τετραγωνική μορφή με αποτέλεσμα να μην είναι αναστρέψιμοι (και κατά συνέπεια, ο μετασχηματισμός δεν είναι γεωμετρικός).

\subsubsection{\en{DDRM-CapsNet: Capsule Network Based on Deep Dynamic Routing Mechanism for Complex Data}}

Στο έργο των \en{Liu et al.} \footnote{Σε ελεύθερη μετάφραση: \textquote{Νευρωνικό Δίκτυο από Κάψουλες Βασισμένο πάνω σε Βαθύ Δυναμικό Μηχανισμό Δρομολόγησης για Σύνθετα Δεδομένα}.} \cite{liu2019ddrm} δοκιμάζεται μια πιο σύνθετη αρχιτεκτονική με περισσότερες παραμέτρους προκειμένου το δίκτυο να ανταποκρίνεται καλύτερα σε πιο σύνθετα σύνολα δεδομένων. Συγκεκριμένα, πειραματίζονται με διάφορες αρχιτεκτονικές που εισάγουν ένα, δύο, τρία ή τέσσερα συνελικτικά επίπεδα πριν το πρώτο επίπεδο από κάψουλες (\en{PrimaryCaps}). Επιπλέον, εισάγουν ένα ακόμα επίπεδο από κάψουλες τύπου \en{DigitCaps} με αποτέλεσμα ο (υπολογιστικά κοστοβόρος) αλγόριθμος δρομολόγησης να πρέπει να εφαρμοστεί δύο φορές κατά ένα πρόσθιο πέρασμα. Ακόμη, αυξάνουν την εκφραστικότητα της κάθε κάψουλας με την αύξηση της διάστασης του διανύσματος χαρακτηριστικών που ενθυλακώνουν.\par

Μετά από πειράματα, κατέληξαν στη βέλτιστη αρχιτεκτονική η οποία περιλαμβάνει τρία συνελικτικά επίπεδα και τρία επίπεδα από κάψουλες με το τελευταίο επίπεδο να έχει κάψουλες με μεγαλύτερα σε μήκος διανύσματα ($D=24$). Χρησιμοποιώντας αυτή τη παραμετροποίηση επιτευχθηκε μεταξύ άλλων ακρίβεια 77.5\% στο σύνολο δεδομένων \en{CIFAR10} και 29.93\% ακρίβεια στο σύνολο \en{CIFAR100} (επιδώσεις βελτιωμένες κατά 11\% και 8\% αντοίστηχα σε σχέση με το \cite{sabour2017dynamic}). Βέβαια, όλα αυτά γίνονται με αυξημένο υπολογιστικό κόστος και χωρίς να δοκιμάζεται αν συνεχίζουν να τηρούνται οι βασικές υποθέσεις των νευρωνικών δικτύων με κάψουλες.

\subsubsection{\en{DeepCapsnet: Going Deeper with Capsule Networks}}

Στο έργο των \en{Rajasegaran et al.} \footnote{Σε ελεύθερη μετάφραση: \textquote{Πηγαίνωντας Βαθύτερα με τα Νευρωνικά Δίκτυα από Κάψουλες}.} \cite{rajasegaran2019deepcaps} εισάγεται μια νέα αρχιτεκτονική προκειμένου να ανταποκρίνεται καλύτερα σε πιο σύνθετα δεδομένα. Η αρχιτεκτονική αυτή χρησιμοποιεί υπολειματικές συνδέσεις (\en{residual connections}), ένα δίκτυο αποκωδικοποιητή ως μέθοδο ομαλοποίησης (\en{regularization}) και αποφυγής υπερπροσαρμογής (\en{overfitting}) που δέχεται μόνο το διάνυσμα της προβλεφθήσας κλάσης και ένα δυναμικό αλγόριθμο δρομολόγησης εμπνευσμένο από τρισδιάστατες συνελίξεις (\en{3D convolutions}). \par

Τα πειραματικά αποτελέσματα δείχνουν ότι το μοντέλο, συγκρινόμενο -μεταύ άλλων- με το έργο \cite{sabour2017dynamic} επιτυγχάνει ελαφρώς καλύτερα αποτελέσματα στα δεδομένων (\en{CIFAR10, SVHN} και \en{FashionMNIST}) με λιγότερες παραμέτρους (λόγω του διαμοιρασμού παραμέτρων που προσφαίρει η συνέλιξη). Επίσης, μειώνουν το υπολογιστικό κόστος του δυναμικού αλγορίθμου δρομολόγησης που αναπτύσσουν μειώνωντας τον αριθμό των επαναλλήψεων δρομολόγησης. Τέλος, παρατηρείται μια συνοχή στο τι αναπαριστά το κάθε στοιχείο του διανύσματος χαρακτηριστικών (των καψουλών του τελευταίου επιπέδου), ανεαρτήτως κλάσης. Για παράδειγμα, το $28^\circ$ στοιχείο του διανύσματος φαίνεται να επηρεάζει την κάθετη επιμήκυνση του ψηφίου, ανεάρτητα από την κάψουλα εξόδου (και τη κλάση του ψηφίου). \par

Αν και ο αλγόριθμος δρομολόγησης και η αρχιτεκτονική του προτεινόμενου δικτύου είναι αρκετά διαφορετική από την βασική υλοποίηση \cite{sabour2017dynamic}, δεν έγιναν πειράματα που να αποδεικνύουν ότι οι βασικές ιδιότητες των νευρωνικών δικτύων με κάψουλες διατηρούνται. Μάλλιστα, στην απλή περίπτωση του συνόλου δεδομένων \en{MNIST}, η επίδωση είναι πιο περιορισμένη.

\subsubsection{\en{FSC\textendash CapsNet: Fractionally\textendash Strided Convolutional Capsule Network for complex data}}

Στο έργο των \en{Liu et al.} \footnote{Σε ελεύθερη μετάφραση: \textquote{Νευρωνικά Δίκτυα με Κάψουλες Κλασματικού Βηματισμού για Σύνθετα Δεδομένα}.} \cite{liu2019fsc} προτείνεται μια νέα αρχιτεκτονική η οποία βελτιώνει τόσο το κύριο μέρος του νευρωνικού δικτύου όσο και τον αποκωδικοποιητή. Πιο συγκεκριμένα, αυξάνεται ο αριθμός των συνελικτικών επιπέδων πριν από το πρώτο επίπεδο από κάψουλες προκειμένου να εξάγονται πιο πλούσια χαρακτηριστικά. Επιπρόσθετα, αναφορικά με τον αποκωδικοποιητή, χρησιμοποιούνται συνελικτικά επίπεδα κλιμακωτού βηματισμού (\en{fractionally\textendash strided convolutional layers}) που αποσκοπούν να βελτιώσουν την ποιότητα των ανακατασκευασμένων εικόνων. \par

Μέσα από πειράματα σε πέντε σύνολα δεδομένων επιλέχθηκε η βέλτιστη παραλλαγή της προτεινόμενης αρχιτεκτονικής η οποία αποτελείται από τρία συνελικτικά επίπεδα (που προηγούνται του πρώτου επιπέδου από κάψουλες) και δύο συνελικτικά επίπεδα κλιμακωτού βηματισμού (\en{fractionally\textendash strided convolutional layers}). Ενδεικτικά για τα πειραματικά αποτελέσματα, στο \en{CIFAR10} επιτεύχθηκε ποσοστό ακρίβειας 77.53\% ενώ στο \en{CIFAR100} το αντίστοιχο ποσοστό ήταν 25.83\%. Ακόμα, παρατηρήθηκε (οπτικά) καλύτερη ανακατασκευή των εικόνων λόγω του βελτιωμένου αποκωδικοποιητή. Δυστυχώς, δεν έγιναν τα απαραίτητα πειράματα για να διασφαλιστεί ότι οι ιδιότητες των νευρωνικών δικτύων με κάψουλες διατηρούνται (\en{invariance on capsule activations, equivariance on capsule poses}).

\subsubsection{\en{Self\textendash Attention Capsule Networks for Object Classification}}

Στο έργο των \en{Hoogi et al.} \footnote{Σε ελεύθερη μετάφραση: \textquote{Νευρωνικά Δίκτυα με Κάψουλες και Μηχανισμό Αυτο\textendash προσοχής για την Ταινόμηση Αντικειμένων}.} \cite{hoogi2019self} παρουσιάζεται μια πρωτότυπη αρχιτεκτονική νευρωνικών δικτύων με κάψουλες η οποία ενσωματώνει μηχανισμό αυτοπροσοχής. Αναλυτικότερα, μεταξύ του συνελικτικού επιπέδου και του πρώτου επιπέδου από κάψουλες παρευρίσκεται ένα επίπεδο που υλοποιεί τον μηχανισμό αυτο\textendash προσοχής, όπως τον περιγράψαμε στην ενότητα \ref{sec:transformers}. Με αυτόν τον τρόπο, βελτιώνεται η αρχιτεκτονική του νευρωνικού δικτύου χωρίς να αυξάνεται σημαντικά η υπολογιστική πολυπλοκότητα του δικτύου.\par

Η προτεινόμενη αρχιτεκτονική δοκιμάστηκε σε έξι σύνολα δεδομένων, τα τρία εκ των οποίων αποτελούν σύνολα από ιατρικές εικόνες. Αναλυτικότερα για το διαδεδομένο σύνολο δεδομένων \en{CIFAR10}, παρατηρήθηκε 3.55\% βελτίωση σε σχέση με την υλοποίηση στο \cite{sabour2017dynamic} ενώ στο σύνολο δεδομένων \en{MNIST} δεν παρατηρήθηκε κάποια αξιοσημείωτη βελτίωση. Είναι σημαντικό να επισημάνουμε ότι το μοντέλο δε δοκιμάστικε σε σύνολα δεδομένων όπως το \en{affNIST} και συνεπώς δεν μπορούμε να εγγυηθούμε τη διατήρηση των θεμελιωδών ιδιοτήτων σχετικά με την ικανότητά τους να γενικεύουν σε νέες οπτικές γωνίες.

\subsubsection{\en{DA\textendash CapsNet: Dual Attention Mechanism Capsule Network}}

Στη μελέτη των \en{Huang W.} και \en{Zhou F.} \footnote{Σε ελεύθερη μετάφραση: \textquote{Νευρωνικά Δίκτυα με Κάψουλες και Διπλό Μηχανισμό Προσοχής}.} \cite{huang2020capsnet} δοκιμάζεται μια αρχιτεκτονική νευρωνικών δικτύων με κάψουλες που περιέχει διπλό μηχνισμό προσοχής. Αναλυτικότερα, εφαρμόζεται μηχανισμός προσοχής τόσο μεταξύ των συνελικτικών επίπέδων και του πρώτου επιπέδου από κάψουλες (ονομάζεται \en{Conv\textendash Attention}) όσο και μεταξύ του πρώτου επιπέδου από κάψουλες (\en{PrimaryCaps}) και του τελευταίου επιπέδου από κάψουλες (ο συγκεκριμένος μηχανισμός διαφέρει από το ν προηγούμενο και ονομάζεται \en{Caps\textendash Attention}). Αναφορικά με τον πρώτο, μοιάζει περισσότερο με ένα επίπεδο προσοχής το οποίο γίνεται στους διαφορετικούς χάρτες χαρακτηριστικών (στα κανάλια) αφού η εξαγωγή των βαρών προσοχής περιλαμβάνει τη μέση συνάθροιση (\en{average pooling}) με πυρήνα το πλάτος και ύψος των χαρτών χαρακτηριστικών. Αναφορικά με το δεύτερο μηχανισμό προσοχής, περιλαμβάνει εκπαιδευόμενα βάρη προσοχής που εφαρμόζονται ανά 10 κάψουλες. Μετά από την εφαρμογή του \en{Caps\textendash Attention}, σχηματίζονται άλλες κάψουλες που συμμετέχουν στον μηχανισμό δυναμικής δρομολόγησης, όπως περιγράφεται στο \cite{sabour2017dynamic}.\par

Ο διπλός μηχανισμός προσοχής που ενσωματώνεται στο προτεινόμενο μοντέλο αποσκοπεί στη βελτίωση της αξίας της πληροφορίας που περιγράφεται από τις κάψουλες, στην ελάττωση της περιττής πληροφορίας και στη βελτίωση της ιεραρχίας των καψουλών. Αποδηκνείεται (με μια διαδικασία που προσομοιάζει \en{ablation study}) ότι ο διπλός μηχανισμός προσοχής οδηγεί σε καλύτερα πειραματικά αποτελέσματα σε σχέση με άλλες παραλλαγές νευρωνικών δικτύων με κάψουλες που χρησιμοποιούν μονό μηχανισμό προσοχής. Το δίκτυο δοκιμάζεται σε έξι σύνολα δεδομένων και παρουσιάζει σε όλα βελτιωμένη απόδοση σε σχέση με τη βασική υλοποίηση του \cite{sabour2017dynamic}. Ενδεικτικά, στα σύνολα δεδομένων \en{MNIST, CIFAR10} και \en{smallNORB} παρατηρούνται ποσοστά ακρίβειας ταξινόμησης 99.53, 85.47 και 98.26 αντίστοιχα. Κλείνοντας, αξίζει να σημειώσουμε πως αν και το μοντέλο δοκιμάστηκε στο σύνολο δεδομένων \en{smallNORB}, δεν εξετάστηκε η ικανότητα γενίκευσης σε νέες οπτικές γωνίες, για τις οποίες δεν έχει εκπαιδευτεί.

\subsubsection{\en{Quick\textendash CapsNet (QCN): A Fast Alternative to Capsule Networks}}

Στη μελέτη των \en{Shiri et al.} \footnote{Σε ελεύθερη μετάφραση: \textquote{Μια Ταχεία Εναλλακτική των Νευρωνικών Δικτύων με Κάψουλες}.} \cite{shiri2020quick} αναγνωρίζεται η αδυναμία των νευρωνικών δικτύων με κάψουλες αναφορικά με την ταχύτητα εκπαίδευσης και πρόβλεψης. Για τον σκοπό αυτό γίνονται μια σειρά από πειράματα για τη βελτίωση του υπολογιστικού κόστους με όσο το δυνατόν μικρότερη επίπτωση στις επιδόσεις ταινόμησης. Αναλυτικότερα, αν εξαιρέσουμε τον (προαιρετικό) αποκωδικοποιητή ανεαρτήτου\textendash κλάσης (\en{class\textendash independent}) που περιλαμβάνει συνελικτικά επίπεδα κλιμακωτού βηματισμού (\en{fractionally\textendash strided convolutional layers}) και την αντικατάσταση του δευτέρου συνελικτικού επιπέδου με ένα πλήρως διασυνδεδεμένο επίπεδο, δεν προτείνεται κάποια άλλη τροποποίηση της αρχιτεκτονικής ή του αλγορίθμου δρομολόγησης επί της βασική υλοποίησης. Με άλλα λόγια, η ελάττωση του υπολογιστικού κόστους βασίζεται στην μεταβολή των υπερπαραμέτρων της βασικής υλοποίησης όπως περιγράφεται στο \cite{sabour2017dynamic}. 

Μετά από την παρατήρηση ότι ο αριθμός των καψουλών επιρεάζει καθοριστικά την πολυπλοκότητα του δικτύου, έγιναν πειράματα σε σύνολα δεδομένων (\en{MNIST, FashionMNIST, SVHN, CIFAR10, affNIST}) ύστερα από τον περιορισμό του αριθμού των καψουλών του πρώτου επιπέδου (\en{PrimaryCaps}) από 1152 που είναι στην βασική υλοποίηση σε 8 είτε 6 είτε 4. Παρατηρήθηκε ότι αν και υπήρξε ελάττωση της ακρίβειας, οι χρόνοι εκπαίδευσης και πρόβλεψης αυξάνονταν σημαντικά. Ενδεικτικά, για το σύνολο δεδομένων \en{MNIST} και για την παραλλαγή που χρησιμοποιεί μόλις έξι πρωταρχικές κάψουλες (\en{PrimaryCaps}) και βελτιωμένο αποκωδικοποιητή, η ακρίβεια έπεσε από 99.47\% σε 99.19\% ενώ η ταχύτητα εκπαίδευσης και πρόβλεψης αυξήθηκε κατά 5.5 φορές. Τέλος, έγιναν πειράματα για να διασφαλιστεί η ευρωστία του προτεινόμενου μοντέλου στους αφινικούς μετασχηματισμους - βασική ιδιότητα των νευρωνικών δικτύων με κάψουλες. Στα πειράματα αυτά, παρατηρήθηκε μια πτώση ακρίβειας ταινόμησης της τάξης του 10\%.

\subsubsection{\en{CapsNet vs CNN: Analysis of the Effects of Varying Feature Arrangement}}

Το έργο των \en{Manogaran et al.} \footnote{Σε ελεύθερη μετάφραση: \textquote{\en{CapsNet vs CNN:} νάλυση της Επίδρασης της Διακύμανσης της Χωρικής Διαρρύθμισης των Χαρακτηριστικών}.} \cite{manogaran2020capsnet} εξετάζει την εγκυρότητα της βασικής υπόθεσης ότι τα νευρωνικά δίκτυα με κάψουλες μπορούν να διαζειρίζονται καλύτερα τις σχέσεις μερών\textendash όλου και δεν υποφέρουν από το \textquote{πρόβλημα του \en{Picasso}}. Για την εξέταση της υπόθεσης αυτής, κατασκευάζονται δύο απλά σύνολα δεδομένων. Το πρώτο αποτελείται από παραλληλόγραμμα και ισόπλευρα τρίγωνα. Το δεύτερο, αποτελείται από συνδειασμούς των δύο απλών αντικειμένων (ή μερών) που προαναφέραμε για την σύνθεση αντικειμένων που είναι βέλη και μη\textendash βέλη (τα μη βέλη προκείπτουν από τυχαίες διατάξεις ενός ορθογώνιου παραλληλογράμου και ενός τριγώνου). Στην συνέχεια, ένα συνελικτικό δίκτυο και ένα νευρωνικό δίκτυο με κάψουλες (όπως περιγράφεται στο έργο \cite{sabour2017dynamic} αφού τροποποιήθηκε κατάλληλα) εκπαιδεύτηκαν στο πρώτο σύνολο δεδομένων. Έτσι, τα συνελικτικά επίπεδα των δύο δικτύων έμαθαν να αναγνωρίζουν τα επιμέρους αντικείμενα (τρίγωνα και ορθογώνια παραλληλόγραμμα). Έπειτα, παγώνοντας (\en{freeze}) τα συνελικτικά επίπεδα, τα δύο δίκτυα μετεκπαιδεύτηκαν στο δεύτερο σύνολο δεδομένων στο οποίο και τελικά εξετάστικαν. Από τα πειράματα προέκυψε ότι τα συνελικτικά δίκτυα μπορούσαν καλύτερα να διακρύνουν τα βέλη από τα μη\textendash βέλη σε σχέση με τα νευρωνικά δίκτυα από κάψουλες. Δηλαδή, μπορούσαν να διαχειρίζονται καλύτερα τις σχέσεις μερών (ορθογώνιων παραλληλογράμων και τριγώνων) και αντικειμένων (βελών). Αυτό είναι μια ένδειξη που κλονίζει μια από τις συνηθισμένες υποθέσεις των νευρωνικών δικτύων από κάψουλες. Παρόλα αυτά, σύμφωνα με τους συγγραφείς, είναι πιθανό το προτέρημα των συνελικτικών νευρωνικών δικτύων να πηγάζει από την απλότητα του συνόλου δεδομένων.


\subsubsection{\en{R\textendash CapsNet: An Improvement of Capsule Network for More Complex Data}}

Το έργο των \en{Luo et al.} \footnote{Σε ελεύθερη μετάφραση: \textquote{Μια Βελτίωση των Νευρωνικών Δικτύων με Κάψουλες για Σύνθετα Δεδομένα}.} \cite{luo2019r} προτείνει μια αποδοτική αρχιτεκτονική νευρωνικού δικτύου με 45\% λιγότερες παραμέτρους από αυτήν που περιγράφεται στο \cite{sabour2017dynamic}. Το επιτυγχάνει αυτό εισάγοντας ένα επιπλέον συνελικτικό επίπεδο και μειώνοντας το μέγεθος των πυρήνων συνέλιξης (\en{convolution kernels}). Αναφορικά με τα πειράματα, δοκιμάζεται στα σύνολα δεδομένων \en{CIFAR10} και \en{FashionMNIST}. Στο τελευταίο, η επίδοση στο σύνολο ελέγχου ανέρχεται στο 93.89\% (έναντι 92.57\% της βασικής υλοποίησης \cite{sabour2017dynamic}). Δυστυχώς, δεν γίνεται λόγος για πειράματα που εξετάζουν την διατήρηση των βασικών υποθέσεων της εν λόγω τεχνολογίας νευρωνικών δικτύων.

\subsubsection{\en{CapsNet Based on Encoder and Decoder for Object Detection}}

Στο έργο των \en{Luo et al.} \footnote{Σε ελεύθερη μετάφραση: \textquote{Νευρωνικό Δίκτυο με Κάψουλες Βασισμένο σε Αρχιτεκτονική Κωδικοποιητή\textendash Αποκωδικοποιητή για την Αναγνώριση Αντικειμένων}.} \cite{luo2020capsnet} εξετάζεται η δυνατότητα μοντελοποίησης της μεταφοράς (\en{translation equivariance}) από νευρωνικά δίκτυα με κάψουλες. Για τον σκοπό αυτό, δίνετα ιδιαίτερη έμφαση στην κατασκευή ενός βελτιωμένου αποκωδικοποιητή (με την χρήση συνελικτικών επιπέδων κλιμακωτού βηματισμού - \en{fractionally\textendash strided convolutional layers}\footnote{Ονομάζονται στον χώρο των νευρωνικών δικτύων και ως \en{deconvolutions} ή και \en{transposed convolutions}}). Αποδηκνείεται μέσα από πειράματα σε ένα ελαφρός τροποποιημένο σύνολο δεδομένων \en{MNIST} - στο οποίο τα ψηφία τοποθετούνται σε τυχαίες θέσεις πάνω σε ένα μεγάλο \textquote{καμβά} - ότι η πληροφορία που κωδικοποιείται στις παραμέτρους στιγμιοτύπου της κάθε κάψουλας του τελευταίου επιπέδου (\en{DiggitCaps}) εμπεριέχει και την θέση του ψηφίου στην εικόνα εισόδου. Οι παρατηρήσεις αυτές, ισχύουν και όταν στην εικόνα εισόδου υπάρχουν πολλαπλά ψηφία. Παρόλα αυτά, οι μεταφορές είναι το μόνο είδος μετασχηματισμού στο οποίο το προτεινόμενο δίκτυο δοκιμάζεται (είδος μετασχηματισμού που και τα συνελικτικά νευρωνικά δίκτυα διαχειρίζονται αποδοτικά).

\subsubsection{\en{Efficient\textendash CapsNet: Capsule Network with Self\textendash Attention Routing}}

Η μελέτη των \en{Mazzia et al.} \footnote{Σε ελεύθερη μετάφραση: \textquote{Νευρωνικό Δίκτυο από Κάψουλες με Μηχανισμό Αυτο\textendash Προσοχής}.} \cite{mazzia2021efficient} αντικαθιστά τον αργό, επαναλληπτικό αργόριθμο δρομολόγησης με έναν μηχανισμό αυτοπροσοχής. Μέσα από μια αποδοτική αρχιτεκτονική με πολύ λιγότερες παραμέτρους (160\en{K}) είναι δυνατή η εξαγωγή ισχυρών αναπαραστάσεων (κωδικοποιημένων στα διανύσματα των καψουλών) και η επίτευξη αξιοσημείωτων αποτελεσμάτων σε μια πληθόρα από σύνολα δεδομένων.\par

 Αναλυτικότερα για την αρχιτεκτονική του προτεινόμενου δικτύου, τα πρώτα επίπεδά του συγκροτούνται από τρία συνελικτικά επίπεδα τα οποία ακολουθούνται από ένα επίπεδο διαχωρίσιμης συνέλιξης κατά βάθος (\en{depthwise separable convolution}). Το επίπεδο αυτό μειώνει σημαντικά τον αριθμό των παραμέτρων και συνθέτει τις πρωταρχικές κάψουλες (\en{PrimaryCaps}). Έπειτα ακολουθεί το τελευταίο επίπεδο από κάψουλες \textquote{ψηφίου} (\en{DigitCaps}). Τα διανύσματα των καψουλών αυτών σχηματίζονται μέσω ενός μηχανισμού αυτο\textendash προσοχής (λεπτομέρειες του μηχανισμού αυτού θα περιγραφούν στο επόμενο κεφάλαιο).\par

 Σε ότι αφορά το πειραματικό μέρος της μελέτης, χρησιμοποιήθεικαν τα σύνολα δεδομένων \en{MNIST, smallNORB} και \en{MultiMNIST}. Σε αυτά, το ποσοστιαίο σφάλμα ελέγχου ήταν αντίστοιχα 0.26\%, 2.54\% και 5.1\% (χωρίς \en{ensemble}). Τα προαναφερθέντα αποτελέσματα είναι στο ίδιο εύρος με αυτά διαορετικών υλοποιήσεων των νευρωνικών δικτύων από κάψουλες αλλά χρησιμοποιώντας πολύ λιγότερες παραμέτρους. Αν και τα σύνολα δεδομένων που χρησιμοποιήκαν πιστοποιούν ότι ορισμένες ιδιότητες της τεχνολογίας διατηρούνται (π.χ. ευρωστία σε επικαλυπτόμενα αντικείμενα), δεν εξερευνάται η ικανότητα γενίκευσης σε νέες οπτικές γωνίες (για τις οποίες το δίκτυο δεν έχει εκπαιδευτεί). Επιπλέον, όπως θα δούμε στην συνέχεια, ο νέος αλγόριθμος δρομολόγησης που προτείνεται δεν παρέχει ανατροφοδότηση από τις κάψουλες γονείς στις κάψουλες παιδιά (\en{top\textendash down feedback}).

 \subsubsection{\en{Capsule Network Performance on Complex Data}}

Στη μελέτη των \en{Xi et al.} \footnote{Σε ελεύθερη μετάφραση: \textquote{Επίδοση Νευρωνικών Δικτύων με Κάψουλες σε Σύνθετα Δεδομένα}.} \cite{xi2017capsule} παρουσιάζονται τα αποτελέσματα πειραμάτων στα σύνολα δεδομένων \en{CIFAR10} και \en{MNIST} όπως διενεργήθηκαν σε μια σειρά από παραλλαγές της βασική υλοποίησης του νευρωνικού δικτύου με κάψουλες \cite{sabour2017dynamic}. Ενδεικτικά, οι αλλαγές αφορούσαν την προσθήκη επιπλέον συνελικτικών επιπέδων, την προσθήκη επιπλέων επιπέδων από κάψουλες, την αύξηση του αριθμού των καψουλών, την αλλαγή των παραμέτρων της συνάρτησης σφάλματος κ.α. Από τους πειραματισμούς προκύπτει ότι η αύηση των συνελικτικών επιπέδων οδηγεί σε μια μικρή βελτίωση της επίδοσης ενώ η αύηση των επιπέδων από κάψουλες έχει το αντίστροφο αποτέλεσμα. Συνολικά, καμία από τις παραλλαγές δεν παρήγαγε αξιοσημείωτα αποτελέσματα.

\subsubsection{\en{Reasearch on Image Classification Based on Capsnet}}

Στο επιστημονικό έργο των \en{Dong Z.} και \en{Lin S.} \footnote{Σε ελεύθερη μετάφραση: \textquote{Έρευνα στην Ταξινόμηση Εικόνων Βασισμένη στα Νευρωνικά Δίκτυα από Κάψουλες}.} \cite{dong2019research} κατασκευάζονται δύο μοντέλα ταινόμησης εικόνων με το πρώτο να είναι βασισμένο στην τεχνολογία των \textquote{παραδοσιακών} συνελικτικών νευρωνικών δικτύων (που ενσωματώνουν επίπεδα συνάθροισης) και ττο δεύτερο να είναι ένα δίκτυο από κάψουλες (παρόμοιο με το \cite{sabour2017dynamic}). Μέσα από τα πειράματα, υποστηρίζει τα προτερήματα του δεύτερου μοντέλου ενώ επισημένει τα μειωνεκτήματά του, όταν το σύνολο δεδομένων γίνειται πιο σύνθετο. 

\subsubsection{\en{An Optimization View on Dynamic Routing Between Capsules}}

Στη μελέτη των \en{Wang D.} και \en{Liu Q.} \footnote{Σε ελεύθερη μετάφραση: \textquote{Μια Βελτίωση του Αλγορίθμου Δυναμικής Δρομολόγησης Μεταξύ Καψουλών}.} \cite{wang2018optimization} επιδιώκεται η χρήση μιας τυπικής, μαθηματικής διατύπωσης για την περιγραφή της στρατιγικής του αλγορίθμου δυναμικής δρομολόγησης μέσω συμφωνίας \cite{sabour2017dynamic}. Πιο αναλυτικά, αποδηκνύει ότι αυτή είναι παρόμοια με την ελαχιστοποίηση του σφάλματος ομαδοποίησης με την χρήση της μεθόδου κανονικοποίησης \en{KL} επί των συντελεστών σύζευξης (\en{minimizing a standard clustering loss with KL regularization on the coupling probabilities}). Επιπλέον, προτείνει έναν νέο αλγόριθμο δρομολόγησης που, σε ένα περιορισμένο πλαίσιο εφαρμογής, εμφανίζει ορισμένα προτερήματα έναντι του βασικού αλγορίθμου δυναμικής δρομολόγησης.


\subsubsection{\en{Analysis of Capsule Network (Capsnet) Architectures and Applications}}

Το έργο των \en{Pande et al.} \footnote{Σε ελεύθερη μετάφραση: \textquote{Ανάλυση των Αρχιτεκτονικών και Εφαρμογών των Νευρωνικών Δικτύων με Κάψουλες}.} \cite{pande2018analysis} επιχειρή να συνοψίσει ορισμένες από τις πρόσφατες εξελίξεις των νευρωνικών δικτύων με κάψουλες αλλά και μερικές εφαρμογές τους. Επιπρόσθετα, κάνει αναφορά σε νέες επιτυχίες των συνελικτικών νευρωνικών δικτύων προκειμένου να εμπνεύσει την επιστημονική μελέτη. Από την περιγραφή των βελτιώσεων της βασικής υλοποίησης της εν λόγο τεχνολογίας γίνεται αντιληπτό το μεγάλο εύρως αυτών. Με άλλα λόγια, διαπιστώνεται ότι οι βελτιώσεις εστιάζουν τόσο στην κατασκευή αποδοτικότερου αλγορίθμου δρομολόγησης και μηχανισμού ανακατασκευής εικόνας όσο και στην ποιοτικότερη εξαγωγή χαρακτηριστικών. 


\subsubsection{\en{A Convolutional Neural Network Based on a Capsule Network with Strong Generalization for Bearing Fault Diagnosis}}

Η μελέτη των \en{Zhu et al.} \footnote{Σε ελεύθερη μετάφραση: \textquote{Ένα Νευρωνικό Δίκτυο με Κάψουλες με Ισχυρή Γενίκευση για τη Διάγνωση Σφαλμάτων σε Έδρανα Μηχανών}.} \cite{zhu2019convolutional} περιγράφει την εφαρμογή των νευρωνικών δικτύων με κάψουλες σε ένα σύστημα εποπτείας της υγείας περιστρεφόμενου εοπλισμού (\en{health condition monitoring system of rotating machinery}). Η καινοτομία της παρούσας εφαρμογής σε ότι αφορά την τεχνολογία των νευρωνικών δικτύων από κάψουλες έγκειται στην ελαφρώς τροποποιημένη αρχιτεκτονική του χρησιμοποιούμενου δικτύου. Αναλυτικότερα, προτείνουν την χρήση ενός μπλόκ έναρξης (\en{inception block}) και επιπλέον χρησιμοποιούν το μήκος του μεγαλύτερου διανύσματος για εργασία παλινδρόμησης (\en{regression task}). Τα πειράματα που διενεργήθηκαν στο πλαίσιο της εφαρμογής αποδηκνύουν ότι οι προτεινόμενες αλλαγές οδηγούν σε καλύτερη γενίκευση.


\subsubsection{\en{Identifying Aggression and Toxicity in Comments using Capsule Network}}

Το έργο των \en{Srivastava et al.} \footnote{Σε ελεύθερη μετάφραση: \textquote{Εντοπίζοντας Επιθετικότητα και Τοξικότητα σε Σχόλια Χρησιμοποιώντας Νευρωνικά Δίκτυα με Κάψουλες}.} \cite{srivastava2018identifying} είναι το μόνο (σύμφωνα με την παρούσα βιβλιογραφική έρευνα) που χρησιμοποιεί νευρωνικά δίκτυα με κάψουλες σε περιβάλλον επεξεργασίας φυσικής γλώσσας. Η αρχιτεκτονική που χρησιμοποιείται έχει την δομή του \cite{sabour2017dynamic} αλλά για την αρχικοποίηση των πρωτεύοντων καψουλών (\en{PrimaryCaps}) αντί για συνελικτικά δίκυα χρησιμοποιείται \en{LSTM}. Το προτεινόμενο δίκτυο εμφάνισε πολύ ενθαρυντικά αποτελέσματα στο σύνολο δεδομένων \en{Kaggle\textendash toxic comment} (98.46 \en{ROC AUC}).
