\chapter{Βιβλιογραφική Επισκόπηση}
\label{chap:related_work}

Πριν την έναρξη της εκπόνησης του πρακτικού τμήματος της παρούσας διπλωματικής πραγματοποιήθηκε βιβλιογραφική επισκόπηση προκειμένου να αναζητηθούν εργασίες σχετικές με το θέμα των νευρωνικών δικτύων με κάψουλες. Στο κεφάλαιο αυτό, θα γίνει αναφορά στις σημαντικότερες από αυτές οι οποίες λήφθηκαν υπόψην και ενέμπνευσαν τις μεθόδους που θα αναλύσουμε στο επόμενο κεφάλαιο.\par

Αρχικά, θα παρουσιάσουμε τις τρείς βασικές δημοσιεύσεις των \en{Hinton G. et al.} που θεμελίωσαν τη θεωρία πίσω από τα νευρωνικά δίκτυα με κάψουλες σε ένα πλαίσιο επιβλεπόμενης μάθησης. Έπειτα, θα αναφερθούμε στις ποικίλες παραλλαγές αυτών, όπως προκύπτουν από την τροποποίηση της αρχιτεκτονικής ή του αλγορίθμου δρομολόγησης. Στη συνέχεια, θα γίνει λόγος για τα νευρωνικά δίκτυα με κάψουλες σε περιβάλλον μη\textendash επιβλεπόμενης μάθησης. Τέλος, θα αναλυθούν συνοπτικά εργασίες οι οποίες επιλύουν αποτελεσματικά το γενικκότερο πρόβλημα της γενίκευσης σε νέες οπτικές γωνίες χρησιμοποιώντας αρχιτεκτονικές διαφορετικές από αυτή των νευρωνικών δικτύων με κάψουλες.\par

\section{Θεμελίωση Θεωρίας Νευρωνικών Δικτύων με Κάψουλες}

Όπως έχουμε αναφέρει, η ιδέα των νευρωνικών δικτύων με κάψουλες δεν είναι καινούρια αφού παρουσιάστηκε για πρώτη φορά από τους \en{Hinton G. et al.} το 2011. Παρόλα αυτά, σχετικά πρόσφατα, μετά από διαδοχικές δημοσιεύσεις, ωρίμασε και πέτυχε αξιοσημείωτα αποτελέσματα σε σύνολα δεδομένων όπως το \en{MultiMNIST}\cite{sabour2017dynamic}. Στην ενότητα αυτή θα κάνουμε λόγο για τα πρώτα τρία βασικά έργα πάνω στην εν λόγω αρχιτεκτονική τεχνητών νευρωνικών δικτύων. Πιο αναλυτικά, θα εκινήσουμε από τη δημοσίευση στην οποία πρωτοπαρουσιάστικε η ιδέα και θα καταλήξουμε στην πιο σύνθετη έκδοση των νευρωνικών δικτύων με κάψουλες για επιβλεπόμενη μάθηση που με τις επιδόσεις της στο σύνολο δεδομένων smallNORB\cite{lecun2004learning} κέντρισε το ενδιαφέρον των ερευνητών. 