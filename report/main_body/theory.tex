\chapter{Θεωρητικό Υπόβαθρο}

Στο παρόν κεφάλαιο θα οικοδομήσουμε την απαραίτητη γνώση στην οποία βασίζεται η έρευνα των επόμενων ενοτήτων. Αρχικά, θα παρουσιαστούν συνοπτικά τα τεχνητά νευρωνικά δίκτυα υπό μια μαθηματική σκοπιά. Έπειτα, θα αναλυθούν τα \hyperlink{_capsule_networks}{νευρωνικά δίκτυα με κάψουλες} (\hyperlink{_capsule_networks}{\en{capsule networks}}) τα οποία και αποτελούν το κύριο θέμα της εργασίας. Τέλος, θα γίνει αναφορά σε νέες τεχνικές και αλγορίθμους που χρησιμοποιήθηκαν στο παρόν έργο ώστε η μετέπειτα εισαγωγή των μεθόδων μας για την εξέλιξη των νευρωνικών δικτύων με κάψουλες να είναι περισσότερο ομαλή και κατανοητή.

\section{Τεχνητά Νευρωνικά Δίκτυα}
\subsection{Μηχανική Μάθηση}
Για την επεξήγηση των νευρωνικών δικτύων απαιτείται πρώτα ...
\subsection{Απλά Νευρωνικά Δίκτυα}
\subsection{Συνελικτικά Δίκτυα}
\subsection{Εκπαίδευση Νευρωνικών Δικτύων}
\section{Νευρωνικά Δίκτυα με Κάψουλες}
\section{Μηχανισμός Προσοχής}
\section{Μετασχηματιστές}
\section{Χάρτες Αυτο-οργάνωσης}
\section{Αλγόριθμος \en{EM}}
\section{Variational}