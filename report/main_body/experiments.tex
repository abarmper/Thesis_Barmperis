\chapter{Πειραματική Μελέτη}

Στην παρούσα ενότητα παρουσιάζουμε τα αποτελέσματα των πειραμάτων που διενεργήθηκαν στην κάθε μια οικογένεια αλγορίθμων. Παρόλα αυτά, δεν είναι σκοπός η βελτιστοποίηση της απόδοσης (όπως καταγράφεται από τις επιλεγμένες μετρικές) για κάθε αλγόριθμο. Όπως έχουμε αναφέρει, ο σκοπός της παρούσας διπλωματικής είναι διττός: αφενός επιθυμούμε να εξερευνήσουμε την επίδραση της χαλάρωσης ορισμένων υποθέσεων των νευρωνικών δικτύων με κάψουλες στην απόδοσή τους (μέθοδος 1) και αφετέρου να επιλύσουμε το πρόβλημα της κλιμακωσιμότητας προτείνοντας έναν αποδοτικό αλγόριθμο δρομολόγησης (μέθοδος 3). Η τέταρτη, πολυδύναμη, μέθοδος, βρίσκεται στο μεταίχμιο αυτών με τη δυνατότητα τόσο για επιλεκτική χαλάρωση των περιορισμών της εν λόγω τεχνολογίας όσο για μερική βελτίωση του χρόνου εκπαίδευσης. Τέλος, η δεύτερη μέθοδος, λόγο της μεγάλης υπολογιστικής πολυπλοκότητάς της που δεν επέτρεπε τον εκτενή πειραματισμό, περιορίζεται σε δύο σύνολα δεδομένων και αναλαμβάνει τον σκοπό τις σύγκρισης με τις υπόλοιπες μεθόδους μας.\par

Όπως γίνεται αντιληπτό, κυρίως για τους αλγορίθμους της μεθόδου 1 αλλά και για αυτούς της μεθόδου 4, μας ενδιαφέρει περισσότερο η σχετική επίδοση μεταξύ αυτών αφού αυτή φανερώνει αν οι περιορισμοί που επιβάλλονται τελικά συμβάλουν στην καλύτερη γενίκευση του δικτύου ή όχι. Για τον λόγο αυτό, δίνουμε έμφαση στη σύγκριση των επιδόσεων μεταξύ των αλγορίθμων που ανήκουν στην ίδια οικογένεια (ίδια μέθοδο). Βέβαια, για λόγους πληρότητας, επιλέγουμε τους καλύτερους αλγορίθμους από την κάθε μέθοδο και τους συγκρίνουμε στο τέλος του παρόντος κεφαλαίου.\par

Κρίνεται σκόπιμο να αναφερθεί πως σε κάθε περίπτωση, η πειραματική μας μελέτη δεν είναι πλήρης. Ορισμένοι αλγόριθμοι που αναπτύξαμε (και ειδικά αυτοί που απαντώνται στην πολυδύναμη τέταρτη μέθοδο) διαμορφώνονται από μια πληθώρα υπερπαραμέτρων όπου η κάθε μια επιδρά καθοριστικά στην απόδοσή τους. Επιπρόσθετα, οι μειωμένοι υπολογιστικοί πόροι που διαθέτουμε καθιστούν τη διαδικασία πειραματισμού ιδιαιτέρως χρονοβόρα. Συνεπώς, είναι χρήσιμο να έχουμε υπ' όψη ότι οι επιδώσεις που καταγράφουμε πιθανότατα να επιδέχονται βελτίωση.\par

Το παρόν κεφάλαιο ακολουθεί την εξής διάρθρωση:
\begin{enumerate}
    \item Αρχικά γίνεται μια σύντομη παρουσίαση των συνόλων δεδομένων που χρησιμοποιούμε, των μετρικών αλλά και της πλατφόρμας πειραματισμού.
    \item Έπειτα ακολουθούν οι πειραματικές μελέτες της κάθε μεθόδου ξεχωριστά. Τα περιεχόμενα της κάθε τέτοιας υποενότητας διαφέρουν σημαντικά ανάλογα με τον σκοπό της εκάστοτε μεθόδου. Σε γενικές γραμμές όμως, περιλαμβάνουν τα αποτελέσματα που προκύπτουν από την αναζήτηση ικανοποιητικών υπερπαραμέτρων στα διάφορα σύνολα δεδομένων και τη σύγκριση των αλγορίθμων μεταξύ τους.
    \item Τέλος, επιλέγουμε τους καλύτερους αλγορίθμους από διαφορετικές μεθόδους και τους συγκρίνουμε μεταξύ τους αλλά και με άλλες υλοποιήσεις νευρωνικών δικτύων με κάψουλες που συναντώνται στη βιβλιογραφία.
\end{enumerate}

\section{Πλατφόρμα Διεξαγωγής Πειραμάτων, Μετρικές και Σύνολα Δεδομένων} 
Στην ενότητα αυτή κάνουμε λόγο για τα αμετάβλητα στοιχεία που συνθέτουν το περιβάλλον της πειραματικής μας μελέτης. Αυτά περιλαμβάνουν το υπολογιστικό σύστημα στο οποίο διενεργήθηκαν όλα τα πειράματα, τις μετρικές που χρησιμοποιήθηκαν για την εκτίμηση της επίδοσης και τα σύνολα δεδομένων με τα οποία οι αλγόριθμοι τροφοδοτήθηκαν.
\subsection{Πειραματική Πλατφόρμα}
\subsection{Μετρικές Επίδοσης}
\subsection{Σύνολα Δεδομένων}
\section{Πειραματική Μελέτη Μεθόδου 1}
\section{Πειραματική Μελέτη Μεθόδου 2}
\section{Πειραματική Μελέτη Μεθόδου 3}
\section{Πειραματική Μελέτη Μεθόδου 4}
\section{Σύγκριση Πειραματικών Αποτελεσμάτων Μεθόδων}