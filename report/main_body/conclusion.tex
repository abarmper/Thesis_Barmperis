\chapter{Επίλογος}

\section{Σύνοψη και Συμπεράσματα}

Στην παρούσα διπλωματική εργασία, μελετήσαμε σε βάθος τα νευρωνικά δίκτυα με κάψουλες με σκοπό την κατανόηση των εσωτερικών, κρυφών λειτουργιών τους αλλά και την περαιτέρω βελτίωση της απόδοσής τους προτείνοντας μια κλιμακώσιμη παραλλαγή.\par

Αρχικά, περιγράψαμε με προσιτό τρόπο τον γενικότερο χώρο της μηχανικής μάθησης, ξεκινώντας από την ιδέα της εκμάθησης χαρακτηριστικών και καταλήγοντας στις πιο σύγχρονες αρχιτεκτονικές βαθιών νευρωνικών δικτύων. Φυσικά, αφιερώσαμε μεγάλη έκταση του έργου στην παρουσίαση της κυρίαρχης τεχνολογίας με την οποία καταπιάνεται η παρούσα εργασία, δηλαδή αυτής των νευρωνικών δικτύων με κάψουλες.\par

Μέσα από τη θεωρητική μελέτη, μάθαμε μεταξύ άλλων ότι τα - βιολογικά εμπνευσμένα - νευρωνικά δίκτυα με κάψουλες\footnote{Οι κάψουλες είναι συστάδες νευρικών αποκρίσεων που αναπαριστούν αντικείμενα ή μέρη τους και ενθυλακώνουν ισομεταβλητά χαρακτηριστικά τους όπως η θέση ή ο προσανατολισμός.} αποδομούν τα απεικονιζόμενα αντικείμενα (στιγμιότυπα) μιας εικόνας σε τμήματά τους με τρόπο ώστε να τα αναγνωρίζουν αποδοτικά υπό διαφορετικές οπτικές γωνίες. Επισημάναμε ότι σε ένα τέτοιο δίκτυο, τα χαμηλότερα επίπεδά του περιέχουν κάψουλες που αναπαριστούν απλά τμήματα αντικειμένων. Η μετάβαση σε ανώτερα επίπεδα περιλαμβάνει την ψηφοφορία των καψουλών χαμηλότερου επιπέδου για τις ανώτερες κάψουλες με βάση ποια από αυτές εκτιμούν ότι περιέχει το τμήμα του αντικειμένου που έχουν μάθει να αναπαριστούν. Αυτές οι ψήφοι, μέσω ενός αργού, επαναληπτικού αλγορίθμου δρομολόγησης που εντοπίζει την \textquote{κοινή γνώμη} (δηλαδή τη μεταξύ τους συμφωνία) ενεργοποιούν επιλεκτικά τις κάψουλες επόμενου επιπέδου, συνθέτοντας έτσι μια ιεραρχία μεταξύ μερών και αντικειμένων που ενθαρρύνει την εύρωστη μοντελοποίησή τους.\par

Συνεχίζουμε με τη βιβλιογραφική επισκόπηση στην οποία αναφερθήκαμε σε 30 συνολικά σχετικές με το θέμα εργασίες. Από τη μελέτη τους, εντοπίσαμε τα δύο προβλήματα που αποτελούν και στόχους της εργασίας, δηλαδή, αυτό της αδυναμίας κλιμάκωσης και αυτό της απουσίας πρακτικών αποδείξεων των ισχυρισμών των νευρωνικών δικτύων με κάψουλες. Συμπληρωματικά σημειώνουμε ότι μέσα από το εν λόγω κεφάλαιο διαπιστώσαμε ότι η εισαγωγή λίγων επιπλέον συνελικτικών επιπέδων στην αρχή του νευρωνικού δικτύου βοηθάει στην επίδοση.\par

Μέσα από την πειραματική μελέτη των δημοφιλών συστημάτων που θεμελιώνουν την τεχνολογία, παρατηρήσαμε ότι οι ισχυρισμοί ισχύουν όταν το σύνολο δεδομένων είναι απλό (π.χ. \en{MNIST}). Συγκεκριμένα, μέσω πειραμάτων διαταραχής (\en{perturbation tests}) αποδείξαμε ότι τα νευρωνικά δίκτυα με κάψουλες που περιλαμβάνουν αποκωδικοποιητές με αρκετές παραμέτρους (της τάξης των εκατοντάδων χιλιάδων) πράγματι δημιουργούν εύρωστες εσωτερικές αναπαραστάσεις που μεταβάλλονται ανάλογα με τις αλλαγές στην οπτική γωνία και το στυλ του αντικειμένου. Με πειράματα και σε άλλα σύνολα δεδομένων, φάνηκε ότι η συγκεκριμένη υπόθεση ισχύει ακλόνητα όταν για την εκπαίδευση χρησιμοποιούνται σύνολα δεδομένων που περιέχουν αντικείμενα υπό πολλές βαθμιαίες μεταβολές στις  παραμέτρους στιγμιότυπου των απεικονιζόμενων αντικειμένων. Αντίθετα, δεν μπορούμε να ισχυριστούμε το ίδιο για σύνολα δεδομένων με σύνθετο παρασκήνιο (όπως το \en{CIFAR10}).\par

Μια ακόμα ενδιαφέρουσα υπόθεση που εξετάσαμε είναι αυτή σύμφωνα με την οποία ο αλγόριθμος δρομολόγησης \textquote{φιλτράρει} τις ψήφους ανάλογα με τη συμφωνία μεταξύ τους και επιλέγει την κάψουλα ανώτερου επιπέδου που εκφράζει την \textquote{κοινή γνώμη}. Μέσα από καινοτόμα πειράματα και εργαλεία που φανερώνουν όλες τις εσωτερικές  λειτουργίες του δικτύου, επιβεβαιώσαμε ότι πράγματι, ο αλγόριθμος δρομολόγησης εφαρμόζει φιλτράρισμα βάση (πολυδιάστατης) συμφωνίας. Παρόλα αυτά, η ψηλότερη μέση συμφωνία των ψήφων για μια κάψουλα τελευταίου επιπέδου είδαμε ότι δεν αποτελεί την καλύτερη ένδειξη για το ποια τελικά κάψουλα θα επιλεγεί (ειδικά για πιο σύνθετα σύνολα δεδομένων, όπως το \en{Fashion-MNIST} που δοκιμάσαμε).

Σε ό,τι αφορά το πρόβλημα της κλιμακωσιμότητας, προτείναμε δύο μεθόδους (μέθοδοι 3 και 4) που μέσα από την τροποποίηση του αλγορίθμου δρομολόγησης, αποδεδειγμένα μειώνουν σημαντικά το υπολογιστικό κόστος. Η 4\textsuperscript{η} μέθοδος, αν και επιστημονικά θεμελιωμένη σε μια αποδοτική εκδοχή του αλγορίθμου χαρτών αυτο\textendash οργάνωσης \en{Kohonen}, δεν εμφανίζει ανταγωνιστικά αποτελέσματα υπό τις παραμετροποιήσεις που εξετάσαμε.\par

Αντιθέτως, η 3\textsuperscript{η} μέθοδος που ενσωματώνει έναν μη\textendash επαναληπτικό αλγόριθμο δρομολόγησης εμπνευσμένο από τον δημοφιλή μηχανισμό προσοχής πολλών κεφαλών, εμφανίζει πολύ υποσχόμενα αποτελέσματα σχεδόν σε όλα τα προβλήματα\textendash ορόσημο που δοκιμάστηκε. Για παράδειγμα, με ελάχιστη αναζήτηση στον χώρο των υπερπαραμέτρων και μετά από εκπαίδευση σε μόλις 30 εποχές, επιτυγχάνει μείωση σφάλματος ταξινόμησης κατά 20\% από το αντίστοιχο ποσοστό του δημοφιλούς αλγορίθμου δρομολόγησης \textquote{\en{EM Routing}} στο \en{benchmark} \en{smallNORB} με μόλις το $\frac{1}{10}$ του υπολογιστικού κόστους ανά δευτερόλεπτο (μετρούμενο σε \en{FLOPs}) και με τις μισές παραμέτρους. Επιπρόσθετα, μέσα από πειράματα διαταραχής (\en{perturbation tests}) παρατηρήσαμε ότι πράγματι το δίκτυο που προτείνουμε, όταν εκπαιδεύεται με τη χρήση πλήρως διασυνδεδεμένου ανακατασκευαστή, πραγματοποιεί μια μορφή ανάστροφων γραφικών αφού εξάγει ισομεταβλητά (\en{equivariant}) χαρακτηριστικά των αντικειμένων εισόδου. Σαν παράπλευρα αποτελέσματα, τονίσαμε τη σημασία της χρήσης πλήρως διασυνδεδεμένων ανακατασκευαστών για τα νευρωνικά δίκτυα με κάψουλες ενώ παράλληλα, αναφέραμε ότι η υπόθεση μοναδικού πατέρα (σύμφωνα με την οποία κάθε κάψουλα χαμηλότερου επιπέδου ανήκει μόνο σε μια κάψουλα αμέσως ψηλότερου επιπέδου) δεν οδηγεί πάντα σε καλύτερα αποτελέσματα.\par

Τέλος, πιστοποιούμε ότι η τρίτη μέθοδος πληροί όλες τις χαρακτηριστικές ιδιότητες των νευρωνικών δικτύων με κάψουλες, υποδεικνύοντας έτσι έμπρακτα ότι ο προτεινόμενος αλγόριθμος δρομολόγησης πιθανώς μπορεί να αποτελέσει το δομικό στοιχείο για αποδοτικότερα μεγάλα συστήματα μηχανικής μάθησης.

\section{Μελλοντικές Κατευθύνσεις}
Ευελπιστούμε ότι τα ελκυστικά αποτελέσματα της 3\textsuperscript{ης} μας μεθόδου σε συνδυασμό με την ευνόητη επεξήγηση των νευρωνικών δικτύων με κάψουλες, θα παρακινήσει την ακαδημαϊκή κοινότητα να δώσει προσοχή σε αυτήν την πολλά υποσχόμενη τεχνολογία.\par

Μια άμεση προέκταση που θα μπορούσε να έχει το παρόν έργο είναι η περαιτέρω πειραματική αναζήτηση καλύτερων επιδόσεων της 3\textsuperscript{ης} μεθόδου στον χώρο των υπερπαραμέτρων για όλα τα υποστηριζόμενα σύνολα δεδομένων, ξεχωριστά. Αν και οι περιορισμένοι πόροι που διαθέταμε δε μας επέτρεψαν να διενεργήσουμε επιπλέον πειράματα, είμαστε πεπεισμένοι ότι με ελάχιστες ή καθόλου αλλαγές, η μέθοδος αυτή θα εμφανίσει ακόμα καλύτερα αποτελέσματα.\par

Μια ακόμα σημαντική επέκταση της διπλωματικής αυτής είναι η σύνθεση ενός βαθύτερου συστήματος, χρησιμοποιώντας την τρίτη μέθοδο σαν δομικό στοιχείο και αντλώντας έμπνευση από τις ιδέες του \en{Hinton G.} που παρατίθενται στο \cite{hinton2021represent_GLOM}. Δυστυχώς, αν και υπήρχαν αρκετές ιδέες κλιμάκωσης του έργου, η προσθήκη 5\textsuperscript{ης} μεθόδου ξέφευγε από τα πλαίσια της παρούσας εργασίας.\par

Τέλος, θα μπορούσε να τροποποιηθεί ελαφρώς η τέταρτη μέθοδος ώστε να εφαρμόζεται ο τροποποιημένος αλγόριθμος χαρτών αυτο\textendash οργάνωσης ξεχωριστά για κάθε παράδειγμα εισόδου (και όχι ανά δέσμη εικόνων). Ο πειραματισμός αυτής της μεθόδου σε ένα γρήγορο υπολογιστικό σύστημα με πολλούς επιταχυντές υλικού εκτιμούμε ότι μπορεί να επιφέρει καλύτερα αποτελέσματα.