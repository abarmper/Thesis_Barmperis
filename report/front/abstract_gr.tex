\chapter*{Περίληψη}
\addcontentsline{toc}{chapter}{Περίληψη}



Τελευταία, στον κλάδο της Τεχνητής Νοημοσύνης, παρατηρείται ραγδαία αύξηση του μεγέθους των Βαθιών Νευρωνικών Δικτύων. Με τα νέα συστήματα να έχουν κολοσσιαίο ενεργειακό κόστος για την ανάπτυξή τους, προκύπτει το ερώτημα του αν η απόδοσή τους επιδέχεται βελτίωση.\par

Μια ελπιδοφόρα λύση είναι τα Νευρωνικά Δίκτυα με Κάψουλες που, αντιμετωπίζοντας ανεπάρκειες στον σχεδιασμό της αρχιτεκτονικής των δικτύων, οδηγούν σε συστήματα Όρασης Υπολογιστών με υψηλή απόδοση. Σε αυτά, οι αποκρίσεις τεχνητών νευρώνων του δικτύου οργανώνονται σε ομάδες, τις κάψουλες. Η κάθε κάψουλα μαθαίνει να αναγνωρίζει ένα συγκεκριμένο αντικείμενο (ή τμήμα του). Μέσω μιας διαδικασίας που θυμίζει ανάστροφα γραφικά, αποδομεί το αντικείμενο σε χαρακτηριστικές ιδιότητες όπως η πόζα, τις οποίες ενθυλακώνει. Επειδή σε ένα Νευρωνικό Δίκτυο με Κάψουλες, περιλαμβάνονται πολλά επίπεδα από αυτές, σχηματίζεται μια ιεραρχική διάταξη όπου κάψουλες χαμηλότερων επιπέδων (αναπαριστούν τμήματα αντικειμένου) δρομολογούνται σε ανώτερες κάψουλες που αναγνωρίζουν μεγαλύτερα αντικείμενα και σχηματίζονται από τη σύνθεση μερών τους. Η ιεραρχική αποδόμηση των αντικειμένων μαζί με την εξαγωγή των παραμέτρων πόζας αυτών επιτρέπει την εύρωστη μοντελοποίηση των αντικειμένων από το δίκτυο οδηγώντας σε αποδοτικότερη αναγνώρισή τους υπό διαφορετικές οπτικές γωνίες.\par

Δυστυχώς, τα Νευρωνικά Δίκτυα με Κάψουλες δεν έχουν λάβει τη δέουσα προσοχή, γεγονός που αποδίδεται στη δυσνοητότητά τους και στην αδυναμία κλιμάκωσής τους σε μεγαλύτερα συστήματα. Η αντιμετώπιση αυτών των προβλημάτων αποτελεί τον στόχο της παρούσας εργασίας.\par

Το πρώτο πρόβλημα, το προσεγγίζουμε διατελώντας μια διεξοδική μελέτη στα βασικά έργα που θεμελιώνουν την εν λόγω τεχνολογία. Η μελέτη περιλαμβάνει, μεταξύ άλλων, πρωτότυπα πειράματα που φανερώνουν την εσωτερική λειτουργία της και γραφικά σχήματα που διευκολύνουν την κατανόηση της. Αναφορικά με το δεύτερο πρόβλημα, δανειζόμενοι ιδέες από τον δημοφιλή Μηχανισμό Προσοχής και από τους χάρτες αυτο\textendash οργάνωσης αντίστοιχα, δημιουργούμε δύο νέα, γρήγορα συστήματα Νευρωνικών Δικτύων με Κάψουλες.\par

Μέσα από τα πειράματα, πιστοποιούμε έμπρακτα όλους τους θεμελιακούς ισχυρισμούς της τεχνολογίας που μελετάμε ενώ παράλληλα αποδεικνύουμε ότι το ένα εκ των γρήγορων παραλλαγών που προτείνουμε εμφανίζει την τρίτη καλύτερη επίδοση σε πρόβλημα ορόσημο (\en{smallNORB}) ανοίγοντας τον δρόμο για αποδοτικά, κλιμακώσιμα συστήματα.


% Τα τελευταία χρόνια, στον επιστημονικό κλάδο της Τεχνητής Νοημοσύνης, παρατηρείται ραγδαία αύξηση του μεγέθους των Βαθιών Νευρωνικών Δικτύων. Με τα νέα συστήματα να έχουν ενεργειακό κόστος της τάξεως των εκατομμυρίων για την ανάπτυξη και λειτουργία τους, προκύπτει εύλογα το ερώτημα του κατά πόσον μπορεί να βελτιωθεί η απόδοσή τους.\par

% Μια πολλά υποσχόμενη λύση είναι τα νευρωνικά δίκτυα με κάψουλες που, αντιμετωπίζοντας ανεπάρκειες στον σχεδιασμό της αρχιτεκτονικής των δικτύων, οδηγούν σε συστήματα όρασης υπολογιστών με καλύτερη απόδοση. Πιο αναλυτικά, στα προτεινόμενα συστήματα, οι αποκρίσεις των τεχνητών νευρώνων που απαρτίζουν το δίκτυο οργανώνονται σε ομάδες, τις κάψουλες. Η κάθε κάψουλα μαθαίνει να αναγνωρίζει ένα συγκεκριμένο αντικείμενο (ή τμήμα του). Μέσω μιας διαδικασίας που θυμίζει ανάστροφα γραφικά, αποδομεί το αντικείμενο σε χαρακτηριστικά όπως η πόζα και άλλες ιδιότητές του τις οποίες και ενθυλακώνει. Επειδή σε ένα νευρωνικό δίκτυο με κάψουλες, περιλαμβάνονται πολλά επίπεδα από αυτές, σχηματίζεται μια ιεραρχική διάταξη όπου κάψουλες χαμηλότερων επιπέδων που αναπαριστούν μικρά μέρη ενός αντικειμένου δρομολογούνται σε ανώτερες κάψουλες που αναγνωρίζουν μεγαλύτερα αντικείμενα και σχηματίζονται από τη σύνθεσή τους. Η - βιολογικά εμπνευσμένη - ιεραρχική αποδόμηση των αντικειμένων σε συνδειασμό με την εξαγωγή των παραμέτρων πόζας αυτών επιτρέπει την εύρωστη μοντελοποίηση των αντικειμένων από το δίκτυο οδηγώντας έτσι στην αποδοτικότερη αναγνώρισή τους υπό διαφορετικές οπτικές γωνίες.\par

% Δυστυχώς, τα νευρωνικά δίκτυα με κάψουλες δεν έχουν λάβει τη δέουσα προσοχή, γεγονός που μπορεί να αποδοθεί αφενός στη δυσκολία κατανόησής τους και αφετέρου στην αδυναμία κλιμάκωσής τους σε μεγαλύτερα συστήματα. Η αντιμετώπιση αυτών των δύο προβλημάτων αποτελεί τον στόχο της παρούσας διπλωματικής εργασίας.\par

% Το πρώτο πρόβλημα, το προσεγγίζουμε διατελώντας μια διεξοδική μελέτη στα βασικά έργα που θεμελιώνουν την εν λόγω τεχνολογία. Η μελέτη περιλαμβάνει, μεταξύ άλλων, πρωτότυπα πειράματα που φανερώνουν την εσωτερική λειτουργία της και γραφικά σχήματα που διευκολύνουν την κατανόηση της. Αναφορικά με το δεύτερο πρόβλημα, δανειζόμενοι ιδέες από τον δημοφιλή μηχανισμό προσοχής και από τους χάρτες αυτο\textendash οργάνωσης αντίστοιχα, δημιουργούμε δύο νέα, γρήγορα συστήματα νευρωνικών δικτύων με κάψουλες.\par

% Μέσα από τα πειράματα, πιστοποιούμε έμπρακτα όλους τους θεμελιακούς ισχυρισμούς της τεχνολογίας που μελετάμε ενώ παράλληλα αποδηκνύουμε ότι το ένα εκ των γρήγορων παραλλαγών που προτείνουμε εμφανίζει την τρίτη καλύτερη επίδοση σε πρόβλημα ορόσημο (\en{smallNORB}) ανοίγοντας τον δρόμο για αποδοτικά, κλιμακώσιμα συστήματα.

\section*{Λέξεις κλειδιά}

\noindent
Τεχνητή Νοημοσύνη, Μηχανική Μάθηση, Βαθιά Νευρωνικά Δίκτυα, Όραση Υπολογιστών, Συνελικτικά Νευρωνικά Δίκτυα, Διανυσματικές Κάψουλες, Νευρωνικά Δίκτυα με Κάψουλες, Δυναμική Δρομολόγηση με Συμφωνία, Δρομολόγηση Μεγιστοποίησης Προσδοκιών, Χάρτης Αυτο-οργάνωσης, Μηχανισμός Προσοχής, Κάψουλες με Μηχανισμό Προσοχής Πολλών Κεφαλών, Αλγόριθμος Δρομολόγησης με Προσοχή, Μετασχηματιστές
