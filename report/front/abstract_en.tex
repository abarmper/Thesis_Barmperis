\chapter*{\en{Abstruct}}
\addcontentsline{toc}{chapter}{\en{Abstruct}}
\begin{otherlanguage}{english}
	
Lately, in the field of Artificial Intelligence there is a growing trend for Deeper Neural Networks. Given that these costly Machine Learning systems require extreme amounts of energy resources to be developed, it makes sense to ask whether their efficiency can be improved.\par

Capsule Networks, a modern approach which tackles inefficiencies in core Neural Network's architectural design principles, constitutes a promising solution for building more efficient Computer Vision systems. More specifically, in the proposed systems neural activations are grouped together into \textquote{capsules}. Every capsule is trained to recognize a specific entity (object or object part) and encapsulates it's equivariant properties (e.g. it's pose) which it computes through a procedure that resembles inverse graphics. When feeding a Capsule Network comprised of multiple capsule layers with an image depicting an object, a parse tree is dynamically created as lower level capsules (representing object parts) selectively activate higher level capsules (representing bigger objects) through a routing algorithm. The resulting hierarchical decomposition of the entities along with the extraction of their equivariant properties helps the Network form robust, intrinsic object models thus leading to efficient, viewpoint invariant object recognition.\par

Unfortunately, Capsule Networks have not received much attention due to their abstruse nature as well as their inability to scale into larger systems. Our goal in the thesis at hand is to address those two problems, pushing towards a more sustainable future.\par

The first problem is confronted by our thorough investigation of the fundamental academic articles which define the technology. It also includes extensive, novel experimentation and graphical representations of the results, all aiming to elucidate the new systems. Regarding the second problem, we propose two alternative, fast Capsule Network systems inspired by the popular Attention Mechanism and the algorithm used in Self-Organizing Maps respectively.\par

Through our experiments, we testify the foundational assumptions of Capsule Networks and make significant observations. In addition, we show that one of our proposed systems archives state-of-the-art results on the \en{smallNORB} benchmark while greatly reducing computational load and thus, paving the way for efficient, scalable Capsule Networks.
% Μέσα από τα πειράματα, πιστοποιούμε έμπρακτα όλους τους θεμελιακούς ισχυρισμούς της τεχνολογίας που μελετάμε ενώ παράλληλα αποδηκνύουμε ότι το ένα εκ των γρήγορων παραλλαγών που προτείνουμε εμφανίζει την τρίτη καλύτερη επίδοση σε πρόβλημα ορόσημο (\en{smallNORB}) ανοίγοντας τον δρόμο για αποδοτικά, κλιμακώσιμα συστήματα.
\section*{Keywords}
% \enlargethispage{\baselineskip}
\noindent
Artificial Intelligence, AI, Machine Learning, Deep Neural Networks, Computer Vision,\\\mbox{Convolutional} Networks, Vector Capsules, Matrix Capsules, Capsule Neural Networks, \\\mbox{CapsNet}, Capsule Network, Dynamic Routing by Agreement, Expectation Maximization Routing, Efficient CapsNet, Self Organizing Map, SOM, Kohonen, Competitive Learning, Transformer, Attention, Self-Attention Capsules, RoMAV, Routing by Merged Agreeing Votes, RoWSS, Routing by Winner of Similarity Score, RoWLS, Routing by Winner of Length Scores

\end{otherlanguage}
